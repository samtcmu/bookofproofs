% name: Sam Tetruashvili
% title: preliminaries.tex
% date created: Thu Dec 25 22:14:41 EST 2008
% description: The Preliminaries chapter of The Book of Proofs.

% last modified: Sun Apr 18 10:42:10 EDT 2010

\chapter{Preliminaries}
    \section{Set Operations}
        \begin{definition}
            Let $U$ be the universe of discourse and let $\subs{A, B}{U}$ be sets.
            \begin{itemize}
                \item
                    We say that $A$ is a subset of $B$, denoted $\subs{A}{B}$, if and only
                    if each element of $A$ is an element of $B$, in other words 
                    $\forall a \in A \ a \in B$. We say that $A$ is a strict
                    subset of $B$, denoted $\ssubs{A}{B}$, if and only if $\subs{A}{B}$
                    and $\exists b \in B$ such that $b \not\in A$.
                \item
                    We define the intersection of $A$ and $B$, denoted $\intersect{A}{B}$,
                    to be the set of all elements that are in both $A$ and $B$, i.e.
                    $\intersect{A}{B} = \Set{u \in U \mid u \in A \mbox{ and } u \in B}$.
                \item
                    We define the union of $A$ and $B$, denoted $\union{A}{B}$,
                    to be the set of all elements that are in $A$ or $B$, i.e.
                    $\union{A}{B} = \Set{u \in U \mid u \in A \mbox{ or } u \in B}$.
                \item
                    We define the difference of $A$ and $B$, denoted $\setsub{A}{B}$, as
                    the set of all elements that are in $A$ but not in $B$, i.e.
                    $\setsub{A}{B} = \Set{u \in U \mid u \in A \mbox{ and } u \not\in B}$.
                \item
                    Finally we define the complement of $A$, denoted $\C{A}$, as the
                    set of all elements in the universe that are not in $A$. i.e.
                    $\C{A} = \setsub{U}{A}$.
            \end{itemize}
        \end{definition}
    \section{AGM Inequality}
        \begin{theorem}
            \label{AGM Inequality 1}
            $``\forall x, y \in \set{R} \ 2xy \le x^2 + y^2$ and $xy \le \left(\frac{x + y}{2}\right)^2$;
            furthermore, equality holds only when $x = y$.''
        \end{theorem}
        \begin{proof}
            Let $x, y \in \set{R}$ be given. We first note that squaring a real number
            yields a nonnegative result, namely
            \[
                0 \le (x - y)^2 = x^2 - 2xy + y^2
            \]
            Note that the above inequality becomes an equality only when $x = y$.
            Adding $2xy$ to the inequality above yields
            \begin{equation}
                2xy \le x^2 + y^2
            \end{equation}
            Adding $2xy$ to the inequality above yields
            \begin{equation}
                4xy \le x^2 + 2xy + y^2 = (x + y)^2 \iff xy \le \frac{(x + y)^2}{4}
                                        = \left(\frac{x + y}{2}\right)^2
            \end{equation}
            Thus both inequalities hold and become equalities only when $x = y$. \QED
        \end{proof}
        \begin{corollary}
            $``$The function $f(x) = x(1 - x)$ is maximized when $x = 1 - x$.''
        \end{corollary}
        \begin{proof}
            Let $x \in \set{R}$ be given. By \TheoremRef{AGM Inequality 1} we can conclude
            \[
                x(1 - x) \le \left(\frac{x + 2}{2}\right)^2
            \]
            With equality holding only when $x = 1 - x$. Thus if $x \neq 1 - x$ we know that
            \[
                x(1 - x) <  \left(\frac{x + 2}{2}\right)^2
            \]
            Thus we can maximize $x(1 - x)$ by setting $x = 1 - x$, i.e. $x = \frac{1}{2}$.
            This will yield a maximum value of $f(\frac{1}{2}) = \frac{1}{4}$. \QED
        \end{proof}
        \begin{theorem}
            $``\forall x, y \ge 0 \ \frac{2xy}{x + y} \le \sqrt{xy} \le \frac{x + y}{2}$;
            furthermore equality holds only when $x = y$.''
        \end{theorem}
        \begin{proof}
            Let $x, y \ge 0$ be given. By \TheoremRef{AGM Inequality 1} we can conclude
            \[
                xy \le \left(\frac{x + y}{2}\right)^2
            \]
            Since $x$ and $y$ are both nonnegative we know that both sides of the inequality
            are nonnegative as well. Thus we can safely take the square root of both sides
            and obtain
            \begin{equation}
                \label{AGM 1}
                \sqrt{xy} \le \frac{x + y}{2}
            \end{equation}
            We now multiply both sides of this inequality by $\frac{2\sqrt{xy}}{x + y}$ to
            conclude that
            \begin{equation}
                \label{AGM 2}
                \frac{2xy}{x + y} \le \sqrt{xy}
            \end{equation}
            Combining \equationRef{AGM 1} and \equationRef{AGM 2} we can conclude that
            \[
                \frac{2xy}{x + y} \le \sqrt{xy} \le \frac{x + y}{2}
            \]
            Finally we note that the equalities in this equation hold only when $x = y$
            since they were derived from \TheoremRef{AGM Inequality 1}. \QED
        \end{proof}
    \section{The Principle of Mathematical Induction}
        \begin{theorem}[The Principle of Mathematical Induction]
            $``$Let $\inductive{P}{n}$ be a mathematical statement parametrized by $n$.
            If the following two conditions hold
            \begin{itemize}
                \item
                    $\inductive{P}{1}$ is true.
                \item
                    $\forall k \in \set{N}$ if $\inductive{P}{k}$ is true, then $\inductive{P}{k + 1}$ is true.
            \end{itemize}
            then $\inductive{P}{n}$ is true for all $n \in \set{N}$.''
        \end{theorem}
        \begin{proof}
            Let $\inductive{P}{n}$ be a mathematical statement parametrized by $n$ such that
            \begin{itemize}
                \item
                    $\inductive{P}{1}$ is true.
                \item
                    $\forall k \in \set{N}$ if $\inductive{P}{k}$ is true, then $\inductive{P}{k + 1}$ is true.
            \end{itemize}
            We now proceed by contradiction to show that   
            $\inductive{P}{n}$ is true $\forall n \in \set{N}$. We first let $k \in \set{N}$ be
            given such that $\inductive{P}{k}$ is false and $\forall j < k \
            \inductive{P}{j}$ is true, i.e. choose $k$ to be the smallest natural number for whom
            $\inductive{P}{k}$ is false. We know that such a $k$ exists due to the well ordering
            of the natural numbers. Since we assumed that $\inductive{P}{1}$ is true, we know that
            $k$ must be larger than 1. Thus by construction of $k$ we know that $\inductive{P}{k - 1}$
            is true. By the second assumption we know that $\inductive{P}{k - 1} \then \inductive{P}{k}$
            is true. Thus by modus ponens we can conclude that $\inductive{P}{k}$ is true. 
            This is of course a contradiction; thus $\inductive{P}{n}$ is true for all $n \in \set{N}$
            as required. \QED
        \end{proof}
        \begin{corollary}[The Principle of Strong Mathematical Induction]
            $``$Let $\inductive{P}{n}$ be a mathematical statement.
            If the following two conditions hold
            \begin{itemize}
                \item
                    $\inductive{P}{1}$ is true.
                \item
                    $\forall k \in \set{N}$ if $\forall j \in [k] \ \inductive{P}{j}$ is 
                    true, then $\inductive{P}{k + 1}$ is true.
            \end{itemize}
            then $\inductive{P}{n}$ is true for all $n \in \set{N}$.''
        \end{corollary}
        \begin{proof}
            % TODO
        \end{proof}
        \begin{corollary}
            % TODO
        \end{corollary}
        \begin{proof}
            % TODO
        \end{proof}
    \section{The Geometric Series}
        \begin{definition}
            Let $a, \rho \in \set{R}$ and $n \in \set{N}$. 
            \begin{itemize}
                \item
                    We say that $\dsum_{i = 0}^{\infty} a \cdot \rho^i$ is a geometric series.
                \item
                    We say that $\dsum_{i = 0}^{n} a \cdot \rho^i$ is a finite geometric series.
            \end{itemize}
        \end{definition}
        \begin{theorem}
            $``\forall \rho \in (0, 1) \
            \dsum_{i = 0}^{\infty} \rho^i = \dfrac{1}{1 - \rho}$.''
            \label{Close Form of Geometric Series}
        \end{theorem}
        \begin{proof}
            Let $\rho \in (0, 1)$ be given. Let $X$ be defined as follows
            \begin{equation}
                X = \sum_{i = 0}^{\infty} \rho^i
            \end{equation}
            This implies that $\rho X$ is given as follows
            \begin{equation}
                \rho X = \rho \sum_{i = 0}^{\infty} \rho^i = \sum_{i = 0}^{\infty} \rho^{i + 1} =
                         \sum_{i = 1}^{\infty} \rho^i
            \end{equation}
            We now compute $X - \rho X$ as follows
            \begin{derivation}{=}
                X - \rho X & \dsum_{i = 0}^{\infty} \rho^i - \dsum_{i = 1}^{\infty} \rho^i \\
                           & 1 + \dsum_{i = 1}^{\infty} \rho^i - \dsum_{i = 1}^{\infty} \rho^i \\
                           & 1 + \dsum_{i = 1}^{\infty} (\rho^i - \rho^i) \\
                           & 1
            \end{derivation}
            Thus we have $X - \rho X = 1$, solving for $X$ given us
            \begin{equation}
                X = \frac{1}{1 - \rho}
            \end{equation}
            Thus we can conclude that $\dsum_{i = 0}^{\infty} \rho^i = \dfrac{1}{1 - \rho}$
            as required. \QED
        \end{proof}
        \begin{corollary}
            $``\forall a \in \set{R} \ \forall \rho \in (0, 1) \
            \dsum_{i = 0}^{\infty} a \cdot \rho^i = \dfrac{a}{1 - \rho}$.''
        \end{corollary}
        \begin{proof}
            Let $a \in \set{R}$ and $\rho \in (0, 1)$ be given. We now simply factor
            $a$ from the entire sum and get
            \[
                \sum_{i = 0}^{\infty} a \cdot \rho^i = a \sum_{i = 0}^{\infty} \rho^i
            \]
            We can now apply \TheoremRef{Close Form of Geometric Series} to conclude
            that
            \[
                \sum_{i = 0}^{\infty} a \cdot \rho^i = a \frac{1}{1 - \rho} = \frac{a}{1 - \rho}
            \]
            as required. \QED
        \end{proof}
        \begin{theorem}
            % TODO
        \end{theorem}
        \begin{proof}
            % TODO
        \end{proof}
    \section{Functions}
        \begin{definition}
            Let $A$ and $B$ be sets. We say that $f : A \rightarrow B$ is a function if 
            and only if $\subs{f}{\cross{A}{B}}$ such that
            $\forall a \in A \ \exists ! b \in \set{B}$ such that $(a, b) \in f$.
            When $f$ is a function and $(a, b) \in f$ we write $f(a) = b$. If
            $f: A \rightarrow B$ is a function we define the following sets
            \begin{itemize}
                \item
                    We say that $A$ is the domain of $f$, denoted $\domain{f}$.
                \item
                    We say that $B$ is the target of $f$, denoted $\target{f}$.
                \item
                    We define the image of $f$ as 
                    $\image{f} = \Set{b \in B \mid \exists a \in A \mbox{ such that } f(a) = b}$.
            \end{itemize}
        \end{definition}
        \begin{definition}
            Let $A, B$ be sets and let $f : A \rightarrow B$ be a function.
            \begin{itemize}
                \item
                    (Injectivity) We say that $f$ is injective (one to one) if and only if
                    \[
                        \forall a_1, a_2 \in A \mbox{ if } f(a_1) = f(a_2) \mbox{ then } a_1 = a_2.
                    \]
                \item
                    (Surjectivity) We say that $f$ is surjective (onto) if and only if
                    \[
                        \forall b \in B \ \exists a \in A \mbox{ such that } f(a) = b.
                    \]
                \item
                    (Bijectivity) We that that $f$ is bijective (one to one onto)
                    if and only if $f$ is both injective and surjective.
            \end{itemize}
        \end{definition}
        \begin{definition}
            Let $A, B, C$ be sets and let $g: A \rightarrow B$ and $f: B \rightarrow C$
            be functions. We define the composition of $f$ and $g$, denoted
            $\comp{f}{g}: A \rightarrow C$, as $\forall a \in A \ (\comp{f}{g})(a) = f(g(a))$.
            We can clearly see that $\comp{f}{g}$ is a function. Note that $\comp{g}{f}$
            makes no sense in this context as $\domain{f} \neq \image{g}$.
        \end{definition}
    \section{Equivalence Relations}
        \begin{definition}
            Let $S$ and $T$ be sets. 
            \begin{itemize}
                \item
                    We say $R$ is a relation from $S$ to $T$ if and only if     
                    $\subs{R}{\cross{S}{T}}$.
                \item
                    We say $R$ is a relation on $S$ if and only if  
                    $\subs{R}{\cross{S}{S}}$.
                \item
                    If $R$ is a relation from $S$ to $T$ and $(s, t) \in R$ then
                    we write $s \ R \ t$.
            \end{itemize}
        \end{definition}
        \begin{definition}
            Let $S$ be a set and let $R$ be a relation on $S$. We say $R$ is an
            equivalence relation if and only if $R$ has the following properties
            \begin{itemize}
                \item
                    (Reflexivity) $\forall s \in S \ (s, s) \in R$.
                \item
                    (Symmetry) $\forall s_1, s_2 \in S$ if $(s_1, s_2) \in R$ then
                    $(s_2, s_1) \in R$.
                \item
                    (Transitivity) $\forall s_1, s_2, s_3 \in S$ if $(s_1, s_2), (s_2, s_3) \in R$
                    then $(s_1, s_3) \in R$.
            \end{itemize}
        \end{definition}
        \begin{definition}
            Let $S$ be a set, $R$ be an equivalence relation on $S$, $a \in S$.
            The equivalence class of $a$ with respect to $R$ is 
            $\eqclass{a}{R} = \Set{s \in S \mid (a, s) \in R}$, i.e. the set of elements
            of $S$ related to $a$ by the relation $R$.
        \end{definition}
        \begin{definition}
            Let $S$ be a set and $\subs{C}{\power{S}}$. We say
            $C$ is a partition of $S$ if and only if the following are true.
            \begin{itemize}
                \item
                    (Mutual Exclusion) $\forall A, B \in C$ if $A \neq B$ then $\intersect{A}{B} = \Set{}$.
                \item
                    (Collective Exhaustion) $\dunion_{A \in C} A = S$.
            \end{itemize}
            If $C$ is a finite set and $|C| = k$ then we say $C$ partitions
            $S$ into $k$ pieces.
        \end{definition}
        \begin{theorem}[Equivalence Classes Partition]
            ``Let $S$ be a set. If $R$ is an equivalence relation on $S$, then
            the set of all equivalence classes with respect to $R$ form a 
            partition of $S$.''
        \end{theorem}
        \begin{proof}
            Let $S$ be a set and let $R$ be an equivalence relation on $S$. We will
            now define $C$ to be the set of all equivalence class with respect to $R$
            as follows
            \[
                C = \Set{\eqclass{a}{R} \mid a \in S}
            \]
            We can clearly see that $C$ is some subset of the powerset of $S$.
            We now proceed by showing that $C$ is a partition of $S$ by showing that
            it is both collectively exhaustive and mutually exclusive.
            \begin{itemize}
                \item
                    We proceed by showing that $C$ is mutually exclusive via contradiction.
                    So assume that $C$ is not mutually exclusive, i.e. $\exists a, b \in S$
                    such that $\eqclass{a}{R} \neq \eqclass{b}{R}$ and 
                    $\intersect{\eqclass{a}{R}}{\eqclass{b}{R}} \neq \Set{}$.
                    Thus we now proceed by choosing $c \in \intersect{\eqclass{a}{R}}{\eqclass{b}{R}}$
                    and using it to show that $\eqclass{a}{R} = \eqclass{b}{R}$.
                    \begin{itemize}
                        \item
                            We first show $\subs{\eqclass{a}{R}}{\eqclass{b}{R}}$ as follows.
                            Let $x \in \eqclass{a}{R}$ be given. Since $a, c \in \eqclass{a}{R}$, by 
                            definition of $\eqclass{a}{R}$ we know that $(a, x), (a, c) \in R$.
                            Since $R$ is an equivalence relation we know that $R$ is symmetric and transitive,
                            so we also know that $(x, a) \in R$ and since $(a, c) \in R$ we have $(x, c) \in R$.
                            We now note that since $c \in \eqclass{b}{R}$ we know that $(b, c) \in R$. Now by
                            symmetry of $R$ we have $(c, b) \in R$. We can now invoke transitive on
                            $(x, c)$ and $(c, b)$ to conclude that $(x, b) \in R$. Now by symmetry of $R$ we
                            have $(b, x) \in R$, which of course implies that $x \in \eqclass{b}{R}$ as required.
                        \item
                            The same argument can be used to show that $\subs{\eqclass{b}{R}}{\eqclass{a}{R}}$.
                    \end{itemize}
                    We can now conclude that $\eqclass{a}{S} = \eqclass{b}{S}$ which is of course
                    a contradiction. Thus $C$ is mutually exclusive as required.
                \item
                    We now proceed to show that $C$ collectively exhausts $S$ by showing that
                    $\subs{\dunion_{A \in C} A}{S}$ and $\subs{S}{\dunion_{A \in C} A}$.
                    \begin{itemize}
                        \item
                            We first show that $\subs{\dunion_{A \in C} A}{S}$ as follows. 
                            Let $a \in \dunion_{A \in C} A$ be given. By definition of
                            union there must exists $b \in S$ such that $a \in \eqclass{b}{R}$.
                            Now by definition of $\eqclass{b}{R}$ we know that $a \in S$ as required.
                        \item
                            We now show that $\subs{S}{\dunion_{A \in C} A}$ as follows. 
                            Let $a \in S$ be given. Since $R$ is an equivalence relation we know
                            that it is reflexive, thus $(a, a) \in R$ and
                            $a \in \eqclass{a}{R}$. By definition of $C$ we can conclude that
                            $\eqclass{a}{R} \in C$, so $a \in \dunion_{A \in C} A$ as required.
                    \end{itemize}
                    Thus $C$ collectively exhausts $S$ as required.
            \end{itemize}
            Thus we can conclude that $C$ is a partition of $S$ as required. \QED
        \end{proof}

