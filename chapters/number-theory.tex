% name: Sam Tetruashvili
% title: number-theory.tex
% date created: Tue Dec  2 21:04:54 EST 2008
% description: The Number Theory chapter of The Book of Proofs.

% last modified: Mon Apr 19 14:43:13 EDT 2010

\chapter{Number Theory}
    \section{The Division Algorithm}
        \begin{theorem}
            $``\forall n \in \set{Z} \ \forall b \in \set{N} \ \exists ! q, r \in \set{Z}$
            such that $n = bq + r$ and $0 \le r < b$.''
        \end{theorem}
        \begin{proof}
            Let $n \in \set{Z}$ and $b \in \set{N}$ be given.
            \begin{itemize}
                \item
                    We first prove the existence of such a representation.
                    Let $S$ be defined as follows
                    \[
                        S = \Set{n - bm \mid m \in \set{Z} \mbox{ and } n - bm \ge 0}
                    \]
                    Since $S$ is a set of non-negative integers it has a smallest element,
                    call it $r$. By definition of $S$ we know there is a $q \in \set{Z}$
                    such that $r = n - bq \ge 0$. We now show that $0 \le r < b$ via contradiction.
                    Since by definition of $S$ we know that $r \ge 0$ we can assume that
                    $r \ge b$. This of course implies that $r - b = n - bq - b = n - b(q + 1) \ge 0$,
                    i.e. $r - b \in S$ and $r - b < r = \min(S)$; i.e. we have found an element
                    of $S$ that is smaller than the smallest element in $S$, this is 
                    of course a contradiction. Thus $0 \le r < b$ and $n = bq + r$ as required.
                \item
                    We now show that our representation is unique via contradiction, so 
                    assume there are two different representations, i.e. assume $\exists
                    q, q', r, r' \in \set{Z}$ such that
                    \begin{enumerate}
                        \item
                            $n = bq + r$ and $0 \le r < b$
                        \item
                            $n = bq' + r'$ and $0 \le r' < b$
                        \item
                            $q \neq q'$ or $r \neq r'$
                    \end{enumerate}
                    We now note that $bq + r = bq' + r'$ since they are both equal to $n$. We
                    can now rearrange this equation to conclude that
                    \[
                        r - r' = b(q' - q)
                    \]
                    In other words we know that $r - r'$ must be a multiple of $b$. By construction
                    of $r$ and $r'$ we know that $-b < r - r' < b$. The only multiple of $b$ that falls in
                    this range is 0, thus $r - r' = 0$ and $r = r'$. This of course implies
                    that $b(q' - q) = 0$, i.e. $bq = bq'$. Now since $b \neq 0$ we can conclude that $q = q'$.
                    Thus we have shown that $q = q'$ and $r = r'$, which contradicts condition 3.
                    Thus the representation is unique.
            \end{itemize}
            Thus we can conclude that $n = bq + r, 0 \le r < b$, and 
            that such $q$ and $r$ are unique. \QED
        \end{proof}
    \section{The Euclidean Algorithm}
        \begin{definition}
            Let $a, b \in \set{Z}$, we say $a$ divides $b$, and write $\divides{a}{b}$ 
            if and only if $a, b, \frac{b}{a} \in \set{Z}$. We also define 
            $D_a = \Set{d \in \set{Z} \mbox{ such that } \divides{d}{a}}$,
            to be the set of all divisors of $a$.
        \end{definition}
        \begin{definition}
            Let $a, b \in \set{Z}$, we define the greatest common divisor of $a$ and $b$, 
            denoted $\gcd{a}{b}$, to by the largest integer that divides both $a$ and $b$, 
            i.e. $\gcd{a}{b} = \max(\intersect{D_a}{D_b})$. From this definition one can clearly
            see $\gcd{a}{b} = \gcd{b}{a}$ since set intersection is commutative.
        \end{definition}
        % TODO State this invariant using b mod a instead of b - a, so it fits better
        %      with the Euclidean Algorithm.
        \begin{theorem}
            $``\forall a, b \in \set{Z} \ \gcd{a}{b} = \gcd{a}{b - a}$.''
            \label{EA Basis}
        \end{theorem}
        \begin{proof}
            Let $a, b \in \set{Z}$ be given. Let $D_a$, $D_b$,  and $D_{b - a}$ be the set of all divisors
            of $a, b$, and $b - a$. We now define $W = \intersect{D_a}{D_b}$ to be the set of integers that
            divide both $a$ and $b$ and $V = \intersect{D_a}{D_{b - a}}$ to be the set of integers that
            divide both $a$ and $b - a$. Note that by definition of greatest common divisor we know that
            $\gcd{a}{b} = \max(W)$ and $\gcd{a}{b - a} = \max(V)$. Thus to show that $\gcd{a}{b} =
            \gcd{a}{b - a}$ it suffices to show that $W = V$.
            \begin{itemize}
                \item
                    We will first show that $\subs{W}{V}$, so let $d \in W$ be given. By definition
                    of $W$ we know that $d$ divides both $a$ and $b$. Thus we can conclude that
                    $\frac{a}{d}, \frac{b}{d} \in \set{Z}$. Since the integers are closed under subtraction
                    we can conclude that $\frac{b}{d} - \frac{a}{d} = \frac{b - a}{d} \in \set{Z}$.
                    Thus we can conclude that $d$ must also divides $b - a$. Since we know $d$ divides
                    $a$ and $b - a$ we know that $d \in V$ by definition of $V$.
                \item
                    We will now show that $\subs{V}{W}$. Let $d \in V$ be given. By definition of $V$
                    we know that $d$ must divide both $a$ and $b - a$. Thus we can conclude that
                    $\frac{a}{d}, \frac{b - a}{d} \in \set{Z}$. Since the integers are closed under
                    addition we can conclude that $\frac{b - a}{d} + \frac{a}{d} = \frac{b}{d} \in \set{Z}$.
                    Thus we can conclude that $d$ divides $b$. Since we know $d$ divides both $a$
                    and $b$ we can conclude that $d \in W$ by definition of $W$.
            \end{itemize}
            Thus we can conclude that $W = V$ and $\gcd{a}{b} = \gcd{a}{b - a}$. \QED
        \end{proof}
        % TODO State the Euclidean Algorithm and give C or ML code that implements
        %      both the Euclidean and the Extended Euclidean Algorithm.
        \begin{algorithm}
            Theorem \ref{EA Basis} gives us a tool for designing an algorithm for computing the greatest
            common divisor of two integers. This algorithm is generally credited to Euclid and is thus called
            the Euclidean Algorithm.
            \begin{figure}[H]
                \centering
                $\gcd{a}{b} = \left\{\begin{array}{ll}
                    b & \mbox{if } a = 0 \\
                    \gcd{b\mod a}{a} & \mbox{otherwise}
                \end{array}\right.$
                \caption{The Euclidean Algorithm.}
            \end{figure}
        \end{algorithm}
        \begin{theorem}[Lam\'{e}'s Theorem]
            % TODO
        \end{theorem}
        \begin{proof}
            % TODO
        \end{proof}
    \section{Linear Diophantine Equations}
        \begin{theorem}
            $``\forall a, b \in \set{Z} \ \exists x, y \in \set{Z}$ such that $ax + by = \gcd{a}{b}.$''
            \label{lineardiophantinegcd}
        \end{theorem}
        \begin{proof}
            Without loss of generality we may assume both
            $a$ and $b$ are non-negative since multiplying by $-1$ does not change the factors
            of an integer. We can now proceed by strong induction
            on $n = a + b$ to prove the following statement for each $n \in \set{N}$.
            \[
                \inductive{P}{n} = ``\forall a , b \in \set{N} \ 
                a + b = n \then \exists x, y \in \set{Z} \ \ ax + by = \gcd{a}{b}."
            \]
            \begin{itemize}
                \item
                    Base Case: If $n = 0$, then both $a$ and $b$ must also be 0. Thus $\gcd{a}{b} = 0$,
                    so we can choose $x = y = 1$ and conclude that $ax + by = \gcd{a}{b}$ as required.
                \item
                    Induction Step: Let $k \in \set{N}$ be given; assume  $\forall j \le k$
                    that $\inductive{P}{j}$ is true. Let $a, b \in \set{N}$ be given such
                    that $a + b = k + 1$. Without loss of generality we may assume that $b > a$.
                    Now we note that $\gcd{a}{b} = \gcd{a}{b - a}$. We now have two cases 
                    depending on whether or not $a = 0$.
                    \begin{itemize}
                        \item
                            If $a = 0$ that we know that $\gcd{a}{b} = b$ so we can choose $x = 0$ and $y = 1$
                            and conclude that $ax + by = 0(0) + b(1) = b = \gcd{a}{b}$ as required.
                        \item
                            If $a > 0$ then we know that $b = k + 1 - a \le k$. We now note that $a + (b - a) = b \le k$
                            and $\gcd{a}{b} = \gcd{a}{b - a}$ by the Euclidean Algorithm. Thus we can now invoke
                            $\inductive{P}{b}$ to find $x', y' \in \set{Z}$ such that 
                            $ax' + (b - a)y' = \gcd{a}{b - a}$.
                            This implies that $ax' + (b - a)y' = \gcd{a}{b}$. We can now rearrange the equation
                            as follows
                            \[
                                ax' + (b - a)y' = ax' + by' - ay' = a(x' - y') + by' = \gcd{a}{b}
                            \]
                            So we can now choose $x = x' - y'$ and $y = y'$ and thus conclude that 
                            $ax + by = \gcd{a}{b}$ as required.
                    \end{itemize}
                    Thus in either case we have found $x, y \in \set{Z}$ such that $ax + by = \gcd{a}{b}$.
            \end{itemize}
            Finally by the principle of mathematical induction we can conclude that $\inductive{P}{n}$ is
            true for all $n \in \set{N}$. \QED
        \end{proof}
        % TODO State that we can find the x and y via the Euclidean Algorithm.
        %      In other words state the extended Euclidean Algorithm.
        \begin{corollary}
            $``\forall a, b \in \set{Z} \ \gcd{a}{b} = 1 \then \exists x, y \in \set{Z}$ such
            that $ax + by = 1$''.
            \label{diophantine inverse}
        \end{corollary}
        \begin{definition}
            Let $a, b \in \set{Z}$. We now define the following set
            \begin{itemize}
                \item
                    $S_{a, b} = \Set{(x, y) \in \set{Z}^2 \mid ax + by = \gcd{a}{b}}$, i.e. $S_{a, b}$ is
                    the set of all ordered pairs that solve $ax + by = \gcd{a}{b}$.
            \end{itemize}
        \end{definition}
        \begin{lemma}
            $``\forall a, b \in \set{Z}$ if $\gcd{a}{b} = 1$ and $\exists x_0, y_0 \in \set{Z}$
            such that $ax_0 + by_0 = 1$, then 
            $S_{a, b} = \Set{(x_0 + bk, y_0 - ak) \in \set{Z}^2 \mid k \in \set{Z}}$.''
            \label{Linear Diophantine Equations All Solutions Lemma}
        \end{lemma}
        \begin{proof}
            Let $a, b \in \set{Z}$ be given such that $\gcd{a}{b} = 1$. 
            We can now invoke theorem \ref{lineardiophantinegcd}
            to conclude that $\exists x_0, y_0 \in \set{Z}$ such that $ax_0 + by_0 = \gcd{a}{b} = 1$.
            We now define $T$ as follows
            \[
                T = \Set{(x_0 + bk, y_0 - ak) \in \set{Z}^{2} \mid k \in \set{Z}}
            \]
            We now show that $S_{a, b} = T$ by showing that $\subs{S_{a, b}}{T}$ and $\subs{T}{S_{a, b}}$.
            \begin{itemize}
                \item
                    We first show that $\subs{T}{S_{a, b}}$; so let $(x, y) \in T$ be given.
                    By definition of $T$ we know that $\exists k \in \set{Z}$ such that
                    $x = x_0 + bk$ and $y = y_0 - ak$. We now compute $ax + by$ as follows.
                    \begin{derivation}{=}
                        ax + by & a(x_0 + bk) + b(y_0 - ak) \\
                                & ax_0 + abk + by_0 - abk \\
                                & ax_0 + by_0 \\
                                & \gcd{a}{b} 
                    \end{derivation}
                    Thus we have shown that $(x, y) \in S_{a, b}$ as required.
                \item
                    We now show that $\subs{S_{a, b}}{T}$. Let $(x, y) \in S_{a, b}$ be given.
                    By definition of $S_{a, b}$ we can conclude the following
                    \begin{equation}
                        ax + by = \gcd{a}{b} = ax_0 + by_0
                        \label{LDE All Solutions equation 1}
                    \end{equation}
                    We can now rearrange the equation above as follows
                    \begin{equation}
                        a(x - x_0)=  b(y_0 - y)
                        \label{LDE All Solutions equation 2}
                    \end{equation}
                    Since $\divides{a}{a(x - x_0)}$ we can conclude that $\divides{a}{b(y_0 - y)}$.
                    Now since $a$ and $b$ are relatively prime  we can invoke lemma \ref{FTA lemma 2}
                    to conclude that $\divides{a}{(y_0 - y)}$, i.e. $\dfrac{y_0 - y}{a} \in \set{Z}$. 
                    We can now choose $k = \dfrac{y_0 - y}{a}$ and then rearrange equation 
                    (\ref{LDE All Solutions equation 1}) as follows
                    \[
                        x = x_0 + b\left(\frac{y_0 - y}{a}\right) = x_0 + bk
                    \]
                    To find a similar form for $y$ we can re-arrange (\ref{LDE All Solutions equation 1})
                    as follows
                    \begin{equation}
                        y = y_0 + a\left(\frac{x_0 - x}{b}\right)
                        \label{LDE All Solutions equation 3}
                    \end{equation}
                    We can now divide both sides of equation (\ref{LDE All Solutions equation 2}) by
                    $ab$ to conclude that
                    \[
                        \frac{x - x_0}{b} = \frac{y_0 - y}{a} = k
                    \]
                    We can now multiply the above equation by $-1$ to conclude that
                    \begin{equation}
                        \frac{x_0 - x}{b} = -k 
                        \label{LDE All Solutions equation 4}
                    \end{equation}
                    Thus we can substitute equation (\ref{LDE All Solutions equation 4}) into
                    equation (\ref{LDE All Solutions equation 3}) to conclude that
                    \[
                        y = y_0 - ak
                    \]
                    Thus we have shown that $(x, y) = (x_0 + bk, y_0 - ak)$ when $k = \dfrac{y_0 - y}{a}$,
                    i.e. $(x, y) \in T$ as required.
            \end{itemize} 
           Thus we can conclude that $S_{a, b} = T$ as required. \QED
        \end{proof}
        \begin{theorem}[All Solutions]
            $``\forall a, b \in \set{Z}$ if $\exists x_0, y_0 \in \set{Z}$ such that
            $ax_0 + by_0 = \gcd{a}{b}$, then  $S_{a, b} =
            \Set{\left(x_0 + \dfrac{bk}{\gcd{a}{b}}, y_0 - \dfrac{ak}{\gcd{a}{b}}\right) \in \set{Z}^2 \mid
            k \in \set{Z}}$.''
        \end{theorem}
        \begin{proof}
            Let $a, b \in \set{Z}$ be given. We can now invoke theorem \ref{lineardiophantinegcd}
            to conclude that $\exists x_0, y_0 \in \set{Z}$ such that $ax_0 + by_0 = \gcd{a}{b}$.
            We now define $T$ as follows
            \[
                T = \Set{\left(x_0 + \dfrac{bk}{\gcd{a}{b}}, y_0 - \dfrac{ak}{\gcd{a}{b}}\right) \in \set{Z}^2 
                         \mid k \in \set{Z}}
            \]
            We now let $a' = \dfrac{a}{\gcd{a}{b}}$ and $b' = \dfrac{b}{\gcd{a}{b}}$. Since
            $\gcd{a'}{b'} = 1$ we can invoke lemma \ref{Linear Diophantine Equations All Solutions Lemma} 
            to conclude that $S_{a', b'} = T$. We now finish off the proof by showing that $S_{a, b} = S_{a', b'}$
            as follows.
            \begin{derivation}{\iff}
                (x, y) \in S_{a, b} & ax + by = \gcd{a}{b} & \just{By definition of $S_{a, b}$.}
                                    & \frac{a}{\gcd{a}{b}}x + \frac{b}{\gcd{a}{b}}'y = 1 
                                    & \just{Since $\gcd{a}{b}$ divides $a, b$ and $\gcd{a}{b}$.}
                                    & a'x + b'y = 1 & \just{By definition of $a'$ and $b'$.}
                                    & a'x + b'y = \gcd{a'}{b'} \\
                                    & (x, y) \in S_{a', b'} & \just{By definition of $S_{a', b'}$.}
            \end{derivation}
            Thus we can conclude that $S_{a, b} = S_{a', b'} = T$, which by transitivity of equality
            tells us that $S_{a, b} = T$ as required. \QED
        \end{proof}
        \begin{definition}
            Let $a, b \in \set{Z}$ and $n \in \set{N}$, we define the following
            sets based on these quantities
            \begin{itemize}
                \item
                    $(a, b) = \Set{ax + by \mid x, y \in \set{Z}}$, i.e. $(a, b)$ is the set of
                    all integer combinations of $a$ and $b$.
                \item
                    $\nZ{n} = \Set{nk \mid k \in \set{Z}}$, i.e. $\nZ{n}$ is the set of all
                    multiples of $n$.
            \end{itemize}
        \end{definition}
        \begin{theorem}
            $``\forall a, b \in \set{Z}$ let $n = \gcd{a}{b}$ then $(a, b) = \nZ{n}$.''
        \end{theorem}
        \begin{proof}
            Let $a, b \in \set{Z}$ be given. Let $n = \gcd{a}{b}$. We proceed to show
            that $(a, b) = \nZ{n}$ by showing that $\subs{(a, b)}{\nZ{n}}$ and 
            $\subs{\nZ{n}}{(a, b)}$.
            \begin{itemize}
                \item
                    We first want to show $\subs{\nZ{n}}{(a, b)}$, so let $d \in \nZ{n}$ be
                    given. By definition of $\nZ{n}$ we know $\exists k \in \set{Z}$ such
                    that $d = nk$. By theorem \ref{lineardiophantinegcd} we know $\exists
                    x, y \in \set{Z}$ such that $ax + by = n$. We can simply multiply that
                    equation by $k$ and find that $a(xk) + b(yk) = nk = d$. Thus we can choose
                    $x' = xk$ and $y' = yk$ to find a linear combination of $a$ and $b$ equal to
                    $d$, i.e. $ax' + by' = d$. This of course implies that $d \in (a, b)$.
                \item
                    We now want to show $\subs{(a, b)}{\nZ{n}}$, so let $d \in (a, b)$
                    be given. By definition of $(a, b)$ we know $\exists x, y \in \set{Z}$
                    such that $ax + by = d$. Since $n = \gcd{a}{b}$ we know that $n$ must
                    divide both $a$ and $b$, i.e. $\exists j, k \in \set{Z}$ such that
                    $a = nj$ and $b = nk$. We now rewrite the equation as follows
                    \[
                        ax + by = (nj)x + (nk)y = n(jx + ky) = d
                    \]
                    Thus we have factored $d$ into $nl$ where $l = jx + ky$. Thus we can conclude
                    that $d \in \nZ{n}$.
            \end{itemize}
            Thus we can conclude that $(a, b) = \nZ{n}$. \QED
        \end{proof}
        \begin{corollary}
            $``\forall a, b, n \in \set{Z} \ \exists x, y \in \set{Z}$ such that
            $ax + by = n$ if and only if $\divides{\gcd{a}{b}}{n}$.''
        \end{corollary}
    \section{The Fundamental Theorem of Arithmetic}
        \begin{lemma}
            $``\forall a, n, m \in \set{Z}$ if $\divides{a}{n}$ and $\divides{a}{m}$ then
            $\divides{a}{(n + m)}$.''
            \label{FTA lemma 1}
        \end{lemma}
        \begin{proof}
            Let $a, n, m \in \set{Z}$ be given such that $\divides{a}{n}$ and $\divides{a}{m}$.
            By definition of divisibility we know that $\frac{n}{a}, \frac{m}{a} \in \set{Z}$.
            Now since $\set{Z}$ is closed under addition we can conclude that 
            $\frac{n}{a} + \frac{m}{a} = \frac{n + m}{a} \in \set{Z}$. Thus by definition of
            divisibility we can conclude that $\divides{a}{(n + m)}$ as required.
            \QED
        \end{proof}
        \begin{lemma}
            $``\forall a, b, q \in \set{Z}$ if $\gcd{a}{b} = 1$ and $\divides{a}{qb}$ then
            $\divides{a}{q}$.''
            \label{FTA lemma 2}
        \end{lemma}
        \begin{proof}
            Let $a, b, q \in \set{Z}$ be given such that $\gcd{a}{b} = 1$ and $\divides{a}{qb}$.
            By corollary \ref{diophantine inverse} $\exists x, y \in \set{Z}$ such that
            $ax + by = 1$. If we multiply this equation by $q$ we can conclude that
            $qax + qby = q$. We now note that $\divides{a}{qax}$ since $\frac{qax}{a} = qx \in \set{Z}$
            and $\divides{a}{qby}$ since $\divides{a}{qb}$ by construction. Thus by
            lemma \ref{FTA lemma 1} we can conclude that $\divides{a}{qax + qby}$, which of course
            implies that $\divides{a}{q}$ as required. \QED
        \end{proof}
        \begin{lemma}
            $``\forall n \in \set{N} \ \forall p \in \set{P} \ \forall a_1, \dots, a_n \in \set{Z}$
            if $\divides{p}{\prod_{i = 1}^{n} a_i}$ then $\exists i \in [n]$ such that $\divides{p}{a_i}$.''
            \label{FTA Lemma 3}
        \end{lemma}
        \begin{proof}
            We proceed by induction on $n$ to prove $\inductive{P}{n}$ for all $n \in \set{N}$.
            \[
                \inductive{P}{n} = ``\forall p \in \set{P} \ \forall a_1, \dots, a_n \in \set{Z}
                                    \mbox{ if } \divides{p}{\prod_{i = 1}^{n} a_i} \mbox{ then } 
                                    \exists i \in [n] \mbox{ such that } \divides{p}{a_i}."
            \]
            \begin{itemize}
                \item
                    Base Case: If $n = 1$ then $\inductive{P}{1}$ is trivially true.
                \item
                    Induction Step: Let $k \in \set{N}$ be given; assume $\inductive{P}{k}$ is
                    true. Let $p \in \set{P}$ and $a_1, \dots, a_k, a_{k + 1} \in \set{Z}$ be
                    given such that $\divides{p}{\left(\prod_{i = 1}^{k + 1} a_i\right)}$.
                    We now have two cases either $\divides{p}{a_{k + 1}}$ or $\ndivides{p}{a_{k + 1}}$.
                    \begin{itemize}
                        \item
                            If $\divides{p}{a_{k + 1}}$ then we are done since we have found an
                            $i \in [k + 1]$ such that $\divides{p}{a_i}$.
                        \item
                            If $\ndivides{p}{a_{k + 1}}$ we know that $\gcd{p}{a_{k + 1}} = 1$
                            since $p$ is prime. Since we know that $\divides{p}{a_{k + 1} \prod_{i = 1}^{k} a_i}$
                            we can invoke lemma \ref{FTA lemma 2} to conclude that $\divides{p}{\prod_{i = 1}^k a_i}$.
                            We can now invoke the $\inductive{P}{k}$ to conclude that $\exists i \in [k]$
                            such that $\divides{p}{a_i}$.
                    \end{itemize}
                    In either case we have found an $i \in [k + 1]$ such that $\divides{p}{a_i}$.
                    Thus $\inductive{P}{k + 1}$ is true as required.
            \end{itemize}
            Thus we can conclude that $\inductive{P}{n}$ is true for all $n \in \set{N}$
            as required. \QED
        \end{proof}
        \begin{definition}
            Let $n \in \setsub{\set{Z}}{\Set{0}}$. We say $p_1 \cdots p_k$ is a
            prime factorization of $|n|$ if and only if the following hold.
            \begin{itemize}
                \item
                    $k \in \unionzero{\set{N}}$.
                \item
                    $\forall i \in [k] \ p_i \in \set{P}$; note that
                    the $p_i$'s are not necessarily distinct.
                \item
                    $|n| = \dprod_{i = 1}^{k} p_k$.
            \end{itemize}
            If $n < 0$ we simply prepend a $-$ to the factorization, i.e.
            $n = -p_1 \cdots p_k$. When $n > 0$ and $p_1 \cdots p_k$ is a prime factorization
            of $n$ we write $n = p_1 \cdots p_k$.
        \end{definition}
        \begin{theorem}[The Fundamental Theorem of Arithmetic]
            ``Every non-zero integer has a prime factorization; furthermore this
            factorization is unique up to permutation of the primes in the
            factorization.''
            \label{FTA}
        \end{theorem}
        \begin{proof}
            To prove the Fundamental Theorem of Arithmetic we will proceed by first showing
            that every integer at least one prime factorization, then we will show that this
            factorization is unique. Both steps will be accomplished through induction.
            Throughout the remainder of this proof will will only deal with the positive integers, as
            the prime factorization of $n < 0$ is just the prime factorization of $|n|$ with
            a $-$ prepended to it.
            \begin{enumerate}
                \item
                    Existence: To show that each $n \in \set{N}$ has a prime factorization
                    we will proceed by induction to prove $\inductive{P}{n}$ for all $n \in \set{N}$.
                    \[
                        \inductive{P}{n} = ``\exists k \in \unionzero{\set{N}} \ \exists p_1, \dots, p_k \in \set{P}
                        \mbox{ such that } n = p_1 \cdots p_k."
                    \]
                    \begin{itemize}
                        \item
                            Base Case: If $n = 1$ we just choose $k = 0$ and invoke the
                            convention that the product of zero elements is $1$.
                        \item
                            Induction Step: Let $n \in \set{N}$ be given; assume $\forall j \le n \ 
                            \inductive{P}{j}$ is true. We will now try to find a prime factorization
                            for $n + 1$. We note that $n + 1$ is either prime or composite.
                            \begin{itemize}
                                \item
                                    If $n + 1$ is prime we can just choose $k = 1$ and let $n + 1 = (n + 1)$
                                    be a prime factorization for $n + 1$.
                                \item
                                    If $n + 1$ is composite then
                                    $\exists r, s \in \setsub{[n]}{\Set{1}}$ such that 
                                    \[
                                        n + 1 = rs
                                    \]
                                    We can now invoke $\inductive{P}{r}$ and $\inductive{P}{s}$ to find
                                    $k_1, k_2 \in \unionzero{\set{N}}$ and
                                    $r_1, \dots, r_{k_1}, s_1, \dots, s_{k_2} \in \set{P}$ such that
                                    \[
                                        r = r_1 \cdots r_{k_1} \mbox{ and } s = s_1 \cdots s_{k_2}
                                    \]
                                    are the prime factorizations of $r$ and $s$ respectively. We can now choose
                                    $k = k_1 + k_2$ and construct the prime factorization for $n + 1$
                                    by combining the factorizations for $r$ and $s$ as follows
                                    \[
                                        n + 1 = r_1 \cdots r_{k_1} s_1 \cdots s_{k_2} = p_1 \cdots p_{k}
                                    \]
                            \end{itemize}
                            Thus we have found a prime factorization for $n + 1$ in either case;
                            thus $\inductive{P}{n + 1}$ is true.
                    \end{itemize}
                    Thus we can conclude that $\inductive{P}{n}$ is true for all $n \in \set{N}$
                    as required.
                \item
                    Uniqueness: To show that this factorization is unique up to re-ordering
                    of the $p_i$'s we proceed by induction on $n$ to show that $\inductive{P}{n}$
                    is true for all $n \in \set{N}$.
                    \[
                        \inductive{P}{n} = ``n \mbox{ has a unique prime factorization up to permutation}."
                    \]
                    \begin{itemize}
                        \item
                            Base Case: We know that $\inductive{P}{1}$ is true because we defined
                            the prime factorization of $1$ to be the product of zero integers and
                            we assume that there is only one product of zero integers.
                        \item
                            Induction Step: Let $n \in \set{N}$ be given; assume $\forall j \le n \
                            \inductive{P}{j}$ is true. We will now proceed by contradiction
                            to show that $n + 1$ has a unique prime factorization. Assume $p_1 \cdots p_k$ and
                            $q_1 \cdots q_l$ are two different prime factorizations for $n + 1$, i.e.
                            $k, l \in \unionzero{\set{N}}$, $p_1, \dots, p_k, q_1, \dots, q_l \in \set{P}$,
                            $\exists i \in [k]$ such that $\forall j \in [l] \ p_i \neq q_j$, and
                            \begin{equation}
                                n + 1 = \prod_{i = 1}^{k} p_i = p_1 \prod_{i = 2}^{k} p_i = \prod_{i = 1}^{l} q_i
                                \label{FTA equation 1}
                            \end{equation}
                            Since $n \ge 1$ we know $n + 1 \ge 2$; thus we can conclude that neither
                            $k$ or $l$ are zero since $n + 1 \neq 1$ and the product of zero primes
                            is defined to be 1. Now, since $\divides{p_1}{n + 1}$ we can conclude
                            that $\divides{p_1}{\prod_{i = 1}^{l} q_i}$. We can now invoke lemma
                            \ref{FTA Lemma 3} to conclude $\exists i \in [l]$ such that $\divides{p_1}{q_i}$.
                            Since $q_i$ is prime its only divisors are $1$ and $q_i$. Since $p_1$
                            is also prime we know that $p_1 \neq 1$; thus $p_1 = q_i$.
                            We can now divide equation (\ref{FTA equation 1}) by $p_1$ and then
                            invoke $\inductive{P}{\frac{n + 1}{p_1}}$ to conclude that $\frac{n + 1}{p_1}$
                            has a unique prime factorization. Thus
                            we can conclude that $k = l$ and each of the $p_j$'s in
                            $p_1 \cdots p_k$ appears exactly once as a $q_j$ in $q_1 \cdots q_l$,
                            i.e. they are the same prime factorization up to re-ordering of the factors.
                            This is of course a contradiction, so $\inductive{P}{n + 1}$ is true.
                    \end{itemize}
                    Thus we can conclude that $\inductive{P}{n}$ is true for all $n \in \set{N}$
                    as required.
            \end{enumerate}
            Thus we can conclude that each non-zero integer has a unique prime
            factorization as required. \QED
        \end{proof}
   \section{Modular Arithmetic}
        \begin{definition}
            Let $a, b \in \set{Z}$ and $n \in \set{N}$. We define the congruence relation
            on $\set{Z}$ as follows. We say $a$ is congruent to $b$ modulo
            $n$, written $a \equiv b \mod n$, if and only if $\divides{n}{(a - b)}$.
        \end{definition}
        \begin{lemma}
            $``\forall n \in \set{N}$ congruence modulo $n$ is an equivalence relation.''
            \label{Congruence Equivalence Relation}
        \end{lemma}
        \begin{proof}
            Let $n \in \set{N}$ be given. We will define the congruence modulo $n$ relation 
            on $\set{N}$, denoted by $\sim$ as follows
            \begin{equation}
                \forall a, b \in \set{Z} \ a \sim b \iff a \equiv b \mod n
            \end{equation}
            To show that $\sim$ is an equivalence relation it suffices to show that
            the three properties of equivalence are satisfied.
            \begin{itemize}
                \item
                    Reflexivity: Let $a \in \set{Z}$ be given. We know that $a \equiv a \mod n$
                    since $\divides{n}{a - a}$ since $\divides{n}{0}$. Thus we can conclude that
                    $a \sim a$ by definition of $\sim$.
                \item
                    Symmetry: Let $a, b \in \set{Z}$ be given such that $a \sim b$. By definition
                    of $\sim$ we know that $a \equiv b \mod n$. This implies that $\divides{n}{(a - b)}$
                    and $\frac{a - b}{n} \in \set{Z}$. Since the integer are closed under multiplication
                    we can conclude that $-1 \left(\frac{a - b}{n}\right) = \frac{b - a}{n} \in \set{Z}$.
                    This implies that $\divides{n}{(b - a)}$, which implies that $b \equiv a \mod n$ and
                    that $b \sim a$ as required.
                \item
                    Transitivity: Let $a, b, c \in \set{Z}$ be given such that $a \sim b$ and 
                    $b \sim c$. By definition of $\sim$ we know that $a \equiv b \mod n$ and 
                    $b \equiv c \mod n$; which of course implies that $\divides{n}{(a - b)}$
                    and $\divides{n}{(b - c)}$, i.e. $\frac{a - b}{n}, \frac{b - c}{n} \in \set{Z}$.
                    Since the integers are closed under addition we know that 
                    $\frac{a - b}{n} + \frac{b - c}{n} = \frac{a - c}{n} \in \set{Z}$.
                    This of course implies that $\divides{n}{a - c}, a \equiv c \mod n$, and
                    $a \sim c$ as required.
            \end{itemize}
            Thus congruence modulo $n$ is an equivalence relation. \QED
        \end{proof}
        \begin{lemma}
            $``\forall n \in \set{N} \ \forall a, b, \bar{a}, \bar{b} \in \set{Z}$
            if $a \equiv \bar{a} \mod n$ and $b \equiv \bar{b} \mod n$ then
            $a + b \equiv \bar{a} + \bar{b} \mod n$ and $ab \equiv \bar{a}\bar{b} \mod n$.''
        \end{lemma}
        \begin{proof}
            Let $n \in \set{N}$ and $a, b, \bar{a}, \bar{b} \in \set{Z}$ be given such
            that $a \equiv \bar{a} \mod n$ and $b \equiv \bar{b} \mod n$. By definition 
            of congruence we know that $\divides{n}{(a - \bar{a})}$ and $\divides{n}{(b - \bar{b})}$.
            This of course implies that $\frac{a - \bar{a}}{n}, \frac{b - \bar{b}}{n} \in \set{Z}$.
            Since $\set{Z}$ is closed under addition we can conclude that
            $\frac{a - \bar{a}}{n} + \frac{b - \bar{b}}{n} = \frac{(a + b) - (\bar{a} + \bar{b})}{n} \in \set{Z}$.
            This implies that $\divides{n}{((a + b) - (\bar{a} - \bar{b})}$ by definition of divisibility.
            Now by definition of congruence modulo $n$ we can conclude that 
            \[
                a + b \equiv \bar{a} + \bar{b} \mod n
            \]
            as required.

            We now proceed to show that $ab \equiv \bar{a}\bar{b} \mod n$ using that fact
            that since congruence modulo $n$ is transitive. Since we know that $\frac{a - \bar{a}}{n} \in \set{Z}$
            and that $\set{Z}$ is closed under multiplication, we can conclude that
            $\left(\frac{a - \bar{a}}{n}\right)b = \frac{ab - \bar{a}b}{n} \in \set{Z}$.
            Thus by definition of congruence modulo $n$ we can conclude that
            \begin{equation}
                ab \equiv \bar{a}b \mod n.
                \label{Modular Arithmetic equation 1}
            \end{equation}
            Now since $\frac{b - \bar{b}}{n} \in \set{Z}$, we also know that
            $\bar{a}\left(\frac{b - \bar{b}}{n}\right) = \frac{\bar{a}b - \bar{a}\bar{b}}{n} \in \set{Z}$.
            Thus by definition of congruence modulo $n$ we know that
            \begin{equation}
                \bar{a}b \equiv \bar{a}\bar{b} \mod n
                \label{Modular Arithmetic equation 2}
            \end{equation}
            By lemma \ref{Congruence Equivalence Relation} we know that congruence modulo $n$
            is transitive. Thus we can invoke transitivity of congruence on equations 
            (\ref{Modular Arithmetic equation 1}) and (\ref{Modular Arithmetic equation 2})
            so conclude that
            \[
                ab \equiv \bar{a}\bar{b} \mod n
            \]
            as required. \QED
        \end{proof}
    \section{Infinitely Many Primes}
        \begin{definition}
            We say $n \in \set{Z}$ is prime if and only if the only factors of $n$
            are 1 and $n$. We define $\set{P} = \Set{n \in \set{Z} \mid n \mbox{ is prime}}$,
            i.e. $\set{P}$ is the set of all prime numbers.
        \end{definition}
        \begin{theorem}
            $``$There are infinitely many prime numbers.''
            \label{Infinitely Many Primes}
        \end{theorem}
        \begin{proof}
            We proceed via a proof by contradiction. Assume there are finitely
            many prime numbers, i.e. $\exists n \in \set{N}$ such that $|\set{P}| = n$.
            Moreover assume that $p_1, \dots, p_n$ are the only prime numbers, i.e.
            $\set{P} = \Set{p_1, \dots, p_n}$. We will now define $q \in \set{N}$ as
            follows
            \[
                q = \left(\prod_{i = 1}^n p_i\right) + 1
            \]
            First note that none of the $p_i$'s divide $q$, since $q$ will always by
            congruent to $1 \mod p_i$ by construction. Thus we now have two cases:
            $q$ is either prime or composite.
            \begin{itemize}
                \item
                    If $q$ is prime then we know we have found a new prime number because
                    $q$ cannot be one of the $p_i$'s since $q \equiv 1 \mod p_i$ for any
                    $i$ ranging from 1 to $n$. Thus $q \in \set{P}$ and $|\set{P}| = n + 1$. 
                \item
                    If $q$ is composite, then by the Fundamental Theorem of Arithmetic we
                    can find $r, s \in \set{N}$ such that $r$ is prime and $rs = q$.
                    Note that $q \equiv 0 \mod r$ since $r$ divides $q$. This means
                    that $r$ cannot be one of the $p_i$'s; so we have found a new prime.
                    Thus $r \in \set{P}$ and $|\set{P}| = n + 1$.
            \end{itemize}
            In either case we are forced to conclude that $|\set{P}| = n + 1$, which is of course
            a contradiction. \QED
        \end{proof}
    \section{The Sequence of Prime Numbers}
        \begin{definition}
            We defined the Sequence of Prime Numbers, $p: \set{N} \rightarrow \set{P}$, via
            $p_i = n$ if and only if $n$ is the $i^{th}$ smallest prime number.
        \end{definition}
        \begin{theorem}[The Prime Gap Theorem]
            $``\forall n \in \set{N} \ \exists k \in \set{N} \ p_{k + 1} - p_{k} \ge n$.''
        \end{theorem}
        \begin{proof}
            Let $n \in \set{N}$ be given. We can now proceed by finding a sequence of $n$ consecutive
            natural numbers that are composite. We now choose now consider the following sequence of
            numbers
            \[
                (n + 1)! + 2, (n + 1)! + 3, \dots, (n + 1)! + n, (n + 1)! + (n + 1)
            \]
            We can clearly see that for all $2 \le i \le (n + 1)$ that $(n + 1)! + i$
            is not prime because we can factor $i$ out of the sum. Thus we have a sequence of $n$
            consequtive natural numbers that are composite, we can choose $k \in \set{N}$ such
            that $p_k$ is the largest prime smaller than $(n + 1)! + 1$ and $p_{k + 1}$ is the
            smallest prime larger than $(n + 1)! + (n + 1)$. We know such a $k$ exists since there
            are infintely many primes (theorem \ref{Infinitely Many Primes}). Thus we have found
            $k \in \set{N}$ such that $p_{k + 1} - p_k \ge n$. Thus there are arbitrarily large
            prime gaps as required. \QED
        \end{proof}
    \section{Chinese Remainder Theorem}
        \begin{lemma}
            $``\forall a, b, n \in \set{Z}$ if $\gcd{a}{b} = 1, \divides{a}{n},$ and $\divides{b}{n}$
            then $\divides{ab}{n}$.''
            \label{CRT uniquness lemma}
        \end{lemma}
        \begin{proof}
            Let $a, b, n \in \set{Z}$ be given such that $\gcd{a}{b} = 1, \divides{a}{n},$ and
            $\divides{b}{n}$. By definition of divisibility $\frac{n}{a}, \frac{n}{b} \in \set{Z}$.
            We now let $c = \frac{n}{a}$ and $d = \frac{n}{b}$, i.e. $n = ac = bd$. Since $\divides{a}{n}$
            we know that $\divides{a}{bd}$. Since $\gcd{a}{b} = 1$ we know that $\divides{a}{d}$ by 
            lemma \ref{FTA lemma 2}. Thus by definition of divisibility $\frac{d}{a} = \frac{n}{ab} \in \set{Z}$.
            This of course implies that $\divides{ab}{n}$ as required. \QED
        \end{proof}
        \begin{corollary}
            $\forall k \in \setsub{\set{N}}{\Set{1}} \ \forall a_1, \dots, a_k, n \in \set{Z}$ if
            the $a_i$'s are pairwise relatively prime and $\forall i \in [k] \ \divides{a_i}{n}$ then
            $\divides{\prod_{j = 1}^k a_j}{n}.$''
            \label{CRT uniquness corollary}
        \end{corollary}
        \begin{proof}
            We proceed by induction on $k$ to show $\inductive{P}{k}$ true for all $k \ge 2$.
            \[\begin{array}{r@{\ = \ }c}
                \inductive{P}{k} & ``\mbox{$\forall a_1, \dots, a_k, n \in \set{Z}$ if the $a_i$'s are
                                    pairwise co-prime and $\forall i \in [k] \ \divides{a_i}{n}$} \\
                \multicolumn{1}{l}{} & \mbox{then $\divides{\prod_{j = 1}^{k} a_i}{n}$.''}
            \end{array}\]
            \begin{itemize}
                \item
                    Base Case: If $k = 2$ we know $\inductive{P}{2}$ is true by lemma \ref{CRT uniquness lemma}.
                \item
                    Induction Step: Let $k \in \set{N}$ be given such that $k \ge 2$ and assume
                    $\inductive{P}{k}$ is true. Let $a_1, \dots, a_k, a_{k + 1}, n \in \set{Z}$ be
                    given such that $\forall i, j \in [k + 1]$ if $i \neq j$ then $\gcd{a_i}{a_j} = 1$
                    and $\divides{a_i}{n}$. By $\inductive{P}{k}$ we can conclude that
                    $\divides{\prod_{i = 1}^k a_i}{n}$. We now let $m = \prod_{i = 1}^k a_i$ and note
                    that $\gcd{a_{k + 1}}{m} = 1$. Thus by $\inductive{P}{2}$ we can conclude that
                    $\divides{ma_{k + 1}}{n}$. In other words $\divides{\prod_{i = 1}^{k + 1} a_i}{n}$;
                    thus $\inductive{P}{k + 1}$ is true as required.
            \end{itemize}
            Thus we can conclude that $\inductive{P}{k}$ is true for all $k \ge 2$ as required. \QED
        \end{proof}
        \begin{theorem}
            $``\forall k \in \set{N} \ \forall a_1, \dots, a_k \in \set{Z} \
            \forall n_1, \dots, n_k \in \set{N}$ if  the $n_i$'s are pairwise relatively prime
            then there is a unique $x$ modulo $\prod_{i = 1}^{k} n_i$ such that $\forall i \in [k] \ 
            x \equiv a_i \mod n_i$.''
            \label{Chinese Remainder Theorem}
        \end{theorem}
        \begin{proof}
            Let $k \in \set{N}, a_1, \dots, a_k \in \set{Z}$, and $n_1, \dots, n_k \in \set{N}$ 
            be given such that the $n_i$'s are pairwise relatively prime, i.e. $\forall i, j \in [k]$
            if $i \neq j$ then $\gcd{n_i}{n_j} = 1$. We now proceed by first proving that a 
            solution exists and then showing that the solution is unique. For existence we
            are essentially trying to solve the following system of congruences. 
            \begin{figure}[H]
                \centering
                $\begin{array}{c}
                    x \equiv a_1 \mod n_1 \\
                    \vdots \\
                    x \equiv a_k \mod n_k
                \end{array}$
                \caption{The system of congruences that we are trying to solve.}
                \label{CRT congruences}
            \end{figure}
            We now let $N = \prod_{i = 1}^{k} n_i$ and $N_i = \frac{N}{n_i}$ for each $i \in [k]$. 
            Let $i \in [k]$ be given. We must have $\gcd{N_i}{n_i} = 1$ since $n_i$ is relatively
            prime to every term in the product that represents $N_i$. Thus by corollary 
            \ref{diophantine inverse} $\exists y_i, z_i \in \set{Z}$ such that
            $N_iy_i + n_iz_i = 1$. We can now take this equation modulo $n_i$ and 
            find that $N_iy_i \equiv 1 \mod n_i$. We are now ready to define the solution, $x_0$,
            to the system of congruences in Figure \ref{CRT congruences}.
            \begin{equation}
                x_0 = \sum_{i = 1}^{k} a_iN_iy_i
                \label{CRT solution}
            \end{equation}
            To see that equation (\ref{CRT solution}) solves the system of congruences we proceed
            as follows. Let $i \in [k]$ be given. We now note that $\forall j \in [k]$ if
            $j \neq i$ then $N_j \equiv 0 \mod n_i$ since there is a factor of $n_i$ in the product
            that defines $N_j$. Thus all but the $i^{th}$ term of the sum will be congruent to $0
            \mod n_i$. Thus we can conclude the following
            \[
                x_0 \equiv \sum_{j = 0}^{k} a_jN_jy_j \equiv a_iN_iy_i\mod n_i
            \]
            We now note that $N_iy_i \equiv 1 \mod n_i$ by
            construction of $y_i$. Thus $x_0 \equiv a_i \mod n_i$ as required. Thus we can
            clearly see that $x_0$ is a solution to the system of congruences
            in Figure \ref{CRT congruences}; furthermore we can also see that
            $x = x_0 + Nk$ is also a solution for all $k \in \set{Z}$ since $N \equiv 0 \mod n_i$
            for all $i \in [k]$.
            In other words $x \equiv x_0 \mod N$ is a solution to the system of
            congruences in Figure \ref{CRT congruences}.

            We now proceed via contradiction to show that $x \equiv x_0 \mod N$ is the only solution.
            Assume $\exists y \in \set{Z}$ such that $y \not\equiv x_0 \mod N$ and 
            $\forall i \in [k] \ y \equiv a_i \mod n_i$, i.e. assume $y$ is a different solution. 
            Since we have shown that $x_0$
            is a solution we know that $\forall i \in [k] \ x_0 \equiv a_i \mod n_i$.
            This implies that $\forall i \in [k] \ x_0 \equiv y \mod n_i$. By definition
            of congruence we can conclude that $\forall i \in [k] \ \divides{n_i}{(x_0 - y)}$.
            Since the $n_i$'s are pairwise relatively prime we can invoke corollary
            \ref{CRT uniquness corollary} to conclude that
            $\divides{\left(\prod_{i = 1}^{k} n_i\right)}{(x_0 - y)}$.
            In other words $\divides{N}{(x_0 - y)}$ since $N = \prod_{i = 1}^{k} n_i$.
            Now by definition of congruence we can conclude that $x_0 \equiv y \mod N$.
            This is of course a contradiction.

            Thus $x \equiv x_0 \mod n$ is the only solution to
            the system of congruences in Figure \ref{CRT congruences}. \QED
        \end{proof}
     \section{Fermat's Little Theorem}
        \begin{lemma}
            $``\forall p \in \set{N}$ if $p$ is prime then 
            $\forall k \in [p - 1] \ \binom{p}{k} \equiv 0 \mod p$.''
            \label{prime binomials}
        \end{lemma}
        \begin{proof}
            Let $p \in \set{P}$ and $k \in [p - 1]$ be given. We now note since
            $\binom{p}{k} = \frac{p!}{k!(p - k)!} \in \set{Z}$ $k!(p - k)!$ 
            must divide $p! = p(p - 1)!$. Since $p$ is prime and each term
            of $k!(p - k)!$ is between $1$ and $p - 1$ we can conclude that
            $\gcd{k!(p - k)!}{p} = 1$. Thus we can conclude that $k!(p - k)!$
            must divide $(p - 1)!$. Thus $\frac{\binom{p}{k}}{p} = \frac{(p - 1)!}{k!(p - k)!}
            \in \set{Z}$ and $p$ divides $\binom{p}{k}$. This of course implies that
            $\binom{p}{k} \equiv 0 \mod p$. \QED
        \end{proof}
        \begin{lemma}
            $``\forall p \in \set{N}$ if $p$ is prime then 
            $\forall n \in \set{Z} \ n^p - n \equiv 0 \mod p$.''
            \label{fermat lemma}
        \end{lemma}
        \begin{proof}
            Let $p \in \set{P}$ be given. We proceed by proving the following statement
            for all $n \in \set{N}$ via induction.
            \[
                \inductive{P}{n} = ``n^p - n \equiv 0 \mod p."
            \]
            \begin{itemize}
                \item
                    Base Case: We can clearly see that $\inductive{P}{1}$ is true since
                    $1^p - 1 \equiv 1 - 1 \equiv 0 \mod p$.
                \item
                    Induction Step: Let $k \in \set{N}$ be given; assume $\inductive{P}{k}$
                    is true. We proceed by computing $(k + 1)^p - (k + 1) \mod p$. We now invoke
                    the Binomial Theorem to conclude that
                    \[
                        (k + 1)^p - (k + 1) \equiv \left(\sum_{i = 0}^p \binom{p}{i} k^i 1^{p - i}\right) - (k + 1) \mod p
                    \]
                    We can rewrite the summation as follows
                    \[
                       (k + 1)^p - (k + 1) \equiv \left(\dbinom{p}{0} k^0 + \sum_{i = 1}^{p - 1} \dbinom{p}{i}k^i + \dbinom{p}{p}k^p\right) - (k + 1) \mod p
                    \]
                    We now note by lemma \ref{prime binomials} each term $\sum_{i = 1}^{p - 1} \binom{p}{i}k^i$ 
                    is 0 modulo $p$ since each of the binomial coefficients are 0 modulo $p$. Thus
                    we can now conclude the following
                    \[
                       (k + 1)^p - (k + 1) \equiv (1 + k^p) - (k + 1) \equiv k^p - k \mod p
                    \]
                    By the inductive hypothesis we know that $k^p - k \equiv 0 \mod p$
                    thus we can conclude that that $(k + 1)^p - (k + 1) \equiv 0 \mod p$
                    as required.
            \end{itemize}
            Thus we can conclude that $\inductive{P}{n}$ is true for all $n \in \set{N}$. \QED
        \end{proof}
        \begin{theorem}[Fermat's Little Theorem]
            $``\forall a, p \in \set{Z}$ if $p$ is prime and $\ndivides{p}{a}$, then
            $a^{p - 1} \equiv 1 \mod p$.''
            \label{Fermat's Little Theorem}
        \end{theorem}
        \begin{proof}
            Let $a, p \in \set{Z}$ be given such that $p$ is prime and $\ndivides{p}{a}$.
            Now consider the following $p - 1$ numbers:
            \[
                a, 2a, 3a, \dots, (p - 2)a, (p - 1)a
            \]
            We will first note that $p$ is not a divisor of any of the above numbers. This is true because
            since $p$ is prime we know $\forall i \in [p - 1] \ \ndivides{p}{i}$. Thus $\divides{p}{ia} \iff
            \divides{p}{a}$, but we know by construction of $a$ and $p$ that $\ndivides{p}{a}$ so we can
            conclude that $\ndivides{p}{ia}$.

            We now note that no two different numbers on the above list are in the same equivalence
            class modulo $p$. Without loss of generality let $i, j \in [p - 1]$ be given 
            such that $i > j$. By definition of 
            congruence we know that $ia \equiv ja \mod p \iff \divides{p}{ia - ja}$. We know
            $\divides{p}{ia - ja} \iff \divides{p}{(i - j)a}$, but since $i > j$ we know $(i - j) \in [p - 1]$.
            Thus $(i - j)a$ must be one of the numbers on our list, so $\ndivides{p}{(i - j)a}$.
            This of course implies that $ia \not\equiv ja \mod p$.

            We can now clearly see that each of the non-zero equivalence classes modulo $p$ are
            achieved exactly once by the above numbers. Thus we can conclude the following:
            \[
                a \cdot 2a \cdots (p - 1)a \equiv 1 \cdot 2 \cdots (p - 1) \mod p
            \]
            We can rewrite the congruence relation above as follows:
            \begin{equation}
                a^{p - 1} (p - 1)! \equiv (p - 1)! \mod p 
                \label{fermateq1}
            \end{equation}
            We note that $\gcd{(p - 1)!}{p} = 1$ because since $p$ is prime the only none of the
            natural numbers less that $p$, other than 1, evenly divide it. Thus we can conclude
            $\exists x, y \in \set{Z}$ such that $(p - 1)! \ x + py = 1$. Taking this equation
            modulo $p$ yields the following:
            \[
                (p - 1)! \ x \equiv 1 \mod p
            \]
            In other words $x$ is the inverse of $(p - 1)!$ modulo $p$.
            We can now multiply equation (\ref{fermateq1}) by $x$
            \begin{derivation}{\equiv}
                a^{p - 1} (p - 1)! \ x & (p - 1)! \ x \mod p \\
                a^{p - 1} \cdot 1 & 1 \mod p \\
                a^{p - 1} & 1 \mod p
            \end{derivation}
            Thus we can conclude that $a^{p - 1} \equiv 1 \mod p$ as required. \QED
        \end{proof}
        \begin{proof}[Alternate Proof]
            Let $a, p \in \set{Z}$ be given such that $p$ is prime and $\ndivides{p}{a}$.
            We proceed by invoking lemma \ref{fermat lemma} to conclude
            \[
                a^p - a \equiv 0 \mod p
            \]
            We can now add $a$ to both sides
            \begin{equation}
                a^p \equiv a \mod p
                \label{fermat equation}
            \end{equation}
            Since $\ndivides{p}{a}$ and $p$ is prime we can conclude that $\gcd{a}{p} = 1$,
            thus by theorem \ref{lineardiophantinegcd} $\exists x, y \in \set{Z}$ such that
            $ax + py = 1$. Taking this equation modulo $p$ gives us the following
            \[
                ax \equiv 1 \mod p
            \]
            In other words $x$ is the inverse of $a$ modulo $p$. We can now multiply
            equation \ref{fermat equation} by $x$ and conclude $a^{p - 1} \equiv 1 \mod p$
            as required. \QED
        \end{proof}
        \begin{theorem}[Generalization to Polynomials]
            $``\forall a, p \in \set{Z}$ if $\ndivides{p}{a}$ then
            $p$ is prime if and only if $\forall x \in \set{Z} \ (x + a)^p \equiv x^p + a \mod p$.''
        \end{theorem}
        \begin{proof}
            Let $a, p \in \set{Z}$ be given such that $\ndivides{p}{a}$.
            \begin{itemize}
                \item
                    We first show that if $p$ is prime, then $\forall x \in \set{Z} \
                    (x + a)^p \equiv x^p + a \mod p$. So assume that $p$ is prime and let
                    $x \in \set{Z}$ be given. We now compute $(x + a)^p$ by invoking the 
                    the Binomial Theorem (theorem \ref{Binomial Theorem}) as follows.
                    \begin{derivation}{\equiv}
                        (x + a)^p & \dsum_{k = 0}^{p} \binom{p}{k} x^k a^{p - k} \mod p \\
                                  & a^p + \left( \dsum_{k = 1}^{p - 1} \binom{p}{k} x^k a^{p - k} \right) 
                                    + x^p \mod p
                    \end{derivation}
                    Since $p \in \set{P}$ we can invoke lemma \ref{prime binomials} to conclude that
                    if $1 \le k < p$ then $\binom{p}{k} \equiv 0 \mod p$, thus we can conclude the following.
                    \begin{derivation}{\equiv}
                        (x + a)^p & a^p + \left( \dsum_{k = 1}^{p - 1} (0) x^k a^{p - k} \right) 
                                    + x^p \mod p \\
                                  & a^p + x^p \mod p \\
                                  & x^p + a^p \mod p
                    \end{derivation}
                    We can now invoke Fermat's Little Theorem (theorem \ref{Fermat's Little Theorem})
                    to conclude that $a^p \equiv a \cdot a^{p - 1} \equiv a \mod p$. Thus we can
                    conclude that 
                    \[
                        (x + a)^p \equiv x^n + a \mod p
                    \]
                    as required.
                \item
                    We now show that if $(x + a)^p \equiv x^p + a \mod p$ then $p$ is prime.
                    % TODO
            \end{itemize}
            Thus we can conclude that $p \in \set{P}$ if and only if $(x + a)^n \equiv x^n + a \mod p$. \QED
        \end{proof}
    \section{Euler's Theorem}
        \begin{definition}
            Let $n \in \set{N}$.
            \begin{itemize}
                \item
                    $\Z{n} = \Set{a \in \set{Z} \mid 0 \le a < n}$
                \item
                    $\Zmult{n} = \Set{a \in \set{Z}_{n} \mid \gcd{a}{n} = 1}$
                \item
                    The Euler Totient Function is defined as $\varphi: \set{N} \rightarrow \set{N}$ 
                    such that $\eulerphi{n} = |\Zmult{n}|$.
            \end{itemize}
        \end{definition}
        \begin{fact}
            $\Z{n}$ is a group under modular addition and $\Zmult{n}$ is a group under modular
            multiplication. Furthermore each of these groups is commutative.
        \end{fact}
        \begin{theorem}[Euler's Theorem]
            \label{Euler's Theorem}
            $``\forall n \in \set{N} \ \forall a \in \Zmult{n} \ \ a^{\eulerphi{n}} \equiv 1 \mod n.$''
        \end{theorem}
        \begin{proof}
            Let $n \in \set{N}$ and $a \in \Zmult{n}$ be given. Let $f: \Zmult{n} \rightarrow \Zmult{n}$
            be given via 
            \[
                f(m) = am
            \]
            We now claim that $f$ must be a bijection. We know $f$ is injective because if we take
            $b, c \in \Zmult{n}$ and assume that that $f(b) = f(c)$, i.e. $ab = ac$, then we can
            multiply both sides by the inverse of $a$ modulo $n$, $a^{-1}$, and conclude that $b = c$. We know
            that $f$ is a surjection because if we take $b \in \Zmult{n}$ then 
            $f(a^{-1}b) = aa^{-1}b = b$. In both cases we know that $a^{-1}$ exists because $\Zmult{n}$
            is a group.

            We now consider the product of all the elements $\Zmult{n}$ and use the fact that $f$
            is a bijection to derive the following:
            \[
                \prod_{b \in \Zmult{n}} b = \prod_{b \in \Zmult{n}} f(b) = \prod_{b \in \Zmult{n}} ab
            \]
            Now since the product is independent of $a$ we can factor it out as follows:
            \[
                \prod_{b \in \Zmult{n}} ab = a^{|\Zmult{n}|} \prod_{b \in \Zmult{n}} b
                                           = a^{\eulerphi{n}} \prod_{b \in \Zmult{n}} b
            \]
            We can thus conclude the following:
            \[
                \prod_{b \in \Zmult{n}} b = a^{\eulerphi{n}} \prod_{b \in \Zmult{n}} b
            \]
            Since $\Zmult{n}$ is a group we can multiply both sides of the above equation
            by the inverse of $\left(\prod_{b \in \Zmult{n}} b\right)$ as conclude that $a^{\eulerphi{n}} = 1$.
            Thus we can conclude that $a^{\eulerphi{n}} \equiv 1 \mod n$ as desired. \QED
        \end{proof}
        \begin{proof}[Alternate Proof (Manuel Blum)]
            % TODO Write the proof that uses group theory and considers the subgroup of Z*_n that
            %      is generated by a. This group has order(a, n) many elements and this number
            %      divides phi(n) by Lagrange Theorem. Thus if a^order(a, n) mod n = 1 then so
            %      does a^phi(n) = 1. QED
        \end{proof}
    \section{Wilson's Theorem}
        \begin{lemma}
            $``\forall p \in \set{P} \ \forall i \in \set{Z} \ i^2 \equiv 1 \mod p$ if and only if
            $i \equiv 1 \mod p$ or $i \equiv -1 \mod p$.''
            \label{Wilson's Theorem Lemma 1}
        \end{lemma}
        \begin{proof}
            Let $i, p \in \set{Z}$ be given such that $p$ is prime and $i^2 \equiv 1 \mod p$.
            We now proceed via a sequence of if and only ifs to show that $i \equiv 1 \mod p$ or
            $i \equiv -1 \mod p$.
            \begin{derivation}{\iff}
                i^2 \equiv 1 \mod p & i^2 - 1 \equiv 0 \mod p \\
                                    & (i - 1)(i + 1) \equiv 0 \mod p \\
            \end{derivation}
            Now by definition of modular arithmetic we can conclude that 
            \[
                i^2 \equiv 1 \mod p \iff \divides{p}{(i - 1)(i + 1)}
            \]
            We can now invoke lemma \ref{FTA Lemma 3} to conclude $p$ must divide at least one
            of $i + 1$ and $i - 1$. Also if $p$ divides one of $i + 1$ and $i - 1$ then we also know
            that $p$ divides their product, thus 
            \[
                i^2 \equiv 1 \mod p \iff \divides{p}{(i - 1)} \mbox{ or } \divides{p}{(i + 1)}
            \]
            We can now invoke the definition of modular arithmetic to conclude that
            \begin{derivation}{\iff}
                i^2 \equiv 1 \mod p & (i - 1 \equiv 0 \mod p) \mbox{ or } (i + 1 \equiv 0 \mod p) \\
                                    & (i \equiv 1 \mod p) \mbox{ or } (i \equiv -1 \mod p)
            \end{derivation}
            Thus $i$ is either $1$ or $-1$ modulo $p$ as required. \QED
        \end{proof}
        \begin{lemma}[Uniqueness of Inverse]
            $``\forall p \in \set{P} \ \forall n \in \Zmult{p} \ n$ has a unique inverse in $\Zmult{p}$.''
            \label{Wilson's Theorem Lemma 2}
        \end{lemma}
        \begin{proof}
            Let $p \in \set{P}$ and $n \in \Zmult{p}$ be given. We know that the inverse of $n$
            modulo $p$ exists since $\gcd{n}{p} = 1$, so all we have to do is prove its uniqueness.
            Assume for contradiction that
            the inverse of $n$ in $\Zmult{p}$ is not unique, i.e. assume that $x, y \in \Zmult{p}$
            are inverses of $n$. We now show $x \equiv  y \mod p$ as follows:
            \begin{derivation}{\equiv}
                x & x \cdot 1 \mod p & \\
                  & x \cdot (n \cdot y) \mod p& \just{Since $y$ is the inverse of $n$.}
                  & (x \cdot n) \cdot y \mod p& \just{Since multiplication modulo $p$ is associative.}
                  & 1 \cdot y \mod p & \just{Since $x$ is the inverse of $n$.}
                  & y \mod p
            \end{derivation}
            Thus we can conclude that $n$ has a unique inverse modulo $p$. \QED
        \end{proof}
        \begin{theorem}[Wilson's Theorem]
            ``$\forall p \in \set{N} \ \ p$ is prime if and only if $(p - 1)! \equiv -1 \mod p.$''
            \label{Wilson's Theorem}
        \end{theorem}
        \begin{proof}
            We will proceed by first showing that if $p$ is prime then $(p - 1)! \equiv -1 \mod p$ and then
            showing the converse by proving its contrapositive.

            \begin{itemize}
                \item
                    Let $p \in \set{N}$ be given such that $p$ is prime. 
                    We first note that by Lemma \ref{Wilson's Theorem Lemma 1} $1$ and $p - 1$ are the
                    only elements of $\Zmult{p}$ that are their own inverse modulo $p$. Now consider
                    all the other elements of $\Zmult{p}$. By Lemma \ref{Wilson's Theorem Lemma 2} we know
                    that each of these elements has a unique inverse modulo $p$. This along with the fact
                    that the inverse relation is symmetric tells us the inverse equivalence relation
                    partitions $\setsub{\Zmult{p}}{\Set{1, p - 1}}$ into parts of size 2. Now consider
                    $(p - 1)! \mod p$. Since multiplication modulo $p$ is commutative we can pair up each
                    $x \in \Zmult{p}$ with its unique inverse and cancel it out from the product. This
                    will simplify the product into the following form:
                    \[
                        (p - 1)! \equiv 1 \cdot (p - 1) \equiv p - 1 \equiv -1 \mod p
                    \]
                    Thus $(p - 1)! \equiv -1 \mod p$ as required.
                \item
                    We now show that if $(p - 1)! \equiv -1 \mod p$ then $p$ is prime by proving
                    its contrapositive. So assume that $p$ is composite. This implies there are
                    $a, b \in [p - 1]$ such that $p = ab$. We now have two cases based on whether
                    or not $a$ is equal to $b$.
                    \begin{itemize}
                        \item
                            If $a \neq b$ then we are done since $(p - 1)!$ has both a factor of $a$
                            and $b$ since they are both positive and less then $p$. Thus in this case
                            $(p - 1)! \equiv 0 \mod p$.
                        \item 
                            If $a = b$, then we know that $p = a^2$ so we need to find another factor
                            of $a$. We now have 2 cases based on whether $a = 2$ or $a > 2$. If $a = 2$
                            then we know $p = 4$, one can quickly verify that
                            \[
                                3! \equiv 6 \equiv 2 \not\equiv -1 \mod 4
                            \]
                            as desired. If $a > 2$ then we can find another factor of $a$ amongst
                            the terms of $(p - 1)!$. Since $p = a^2$ and $a > 2$ we know that $p > 2a$.
                            So $2a$ is amongst the terms of $(p - 1)!$. Thus we can conclude that
                            \[
                                (p - 1)! \equiv 0 \mod p
                            \]
                            as required.
                    \end{itemize}
                    Thus we can conclude that if $p$ is not prime then $(p - 1)! \not\equiv -1 \mod p$
                    as required.
            \end{itemize}
            Thus we can conclude that $p$ is prime if and only if $(p - 1)! \equiv -1 \mod p$. \QED
        \end{proof}
    \section{M\"obius Inversion}
        \begin{definition}
            % TODO Define the notion of an arithmetic function.
        \end{definition}
        \begin{theorem}
            ``$\forall n \in \set{N} \ n = \dsum_{\divides{d}{n}} \eulerphi{d}$.''
        \end{theorem}
        \begin{proof}
            Let $n \in \set{N}$ be given. We now proceed by counting that set $[n]$ in two way. Clearly
            $[n]$ has $n$ elements; so we only have to show that $|[n]| = \sum_{\divides{d}{n}} \eulerphi{d}$.
            For each divisor, $d$, of $n$ we define the set $T_d$ as follows
            \[
                T_d = \Set{k \in [n] \mid \gcd{k}{n} = d}
            \]
            We can clearly see that $\Set{T_d}_{\divides{d}{n}}$ partitions $[n]$ since each $k \in [n]$
            has a unique greatest common divisor with $n$. Thus by the Rule of Sum we can conclude that
            \[
                n = \sum_{\divides{d}{n}} |T_d|
            \]
            We now simply have to count the number of elements in each of the partitions. Let $d$ be a
            divisor of $n$. We first note that $T_d \subseteq \Set{kd \ | \ k \in \left[\frac{n}{d}\right]}$,
            since those are the only number less than $n$ that can have a factor of $d$. So now we simply
            need to find all $k \in \left[\frac{n}{d}\right]$ such that $\gcd{kd}{n} = d$. If we divide both
            sides of this equation by $d$ we can conclude that $\gcd{k}{\frac{n}{d}} = 1$. Thus we can conclude
            that $T_d = \Set{kd \ | \ \gcd{k}{\frac{n}{d}} = 1}$, i.e.
            $T_d = \Set{kd \ | \ k \in \Zmult{\frac{n}{d}}}$. We can now clearly see that
            $|T_d| = |\Zmult{\frac{n}{d}}| = \eulerphi{\frac{n}{d}}$. Thus we can conclude the following
            \[
                n = \sum_{\divides{d}{n}} |T_d| = \sum_{\divides{d}{n}} \eulerphi{\frac{n}{d}}
            \]
            We now simple note that
            $\Set{d \in [n] \mid \divides{d}{n}} = \Set{\frac{n}{d} \in [n] \mid \divides{d}{n}}$ and
            conclude that
            \[
                n = \sum_{\divides{d}{n}} \eulerphi{d}
            \]
            as required. \QED
        \end{proof}
        \begin{definition}
            % TODO Define the Mobius Function.
        \end{definition}
        \begin{theorem}
            ``$\forall n \in \set{N} \ \eulerphi{n} = \dsum_{\divides{d}{n}} \mu(n) \frac{n}{d}$.''
        \end{theorem}
        \begin{proof}
            % TODO
        \end{proof}
        \begin{theorem}[M\"obius Inversion Theorem]
            % TODO
        \end{theorem}
        \begin{proof}
            % TODO
        \end{proof}
    \section{The RSA Cryptosystem}
        \begin{definition}
            The \emph{RSA Cryptosystem} is a protocol which allows an entity to recieve
            encrypted messages from another entity. Say Bob would like to be able to recieve
            a message from anyone else in the world. Bob can follow the following protocol
            to allow others to send him messages.
            \begin{itemize}
                \item
                    First Bob has to generate the public and private keys that will
                    be used throughout the protocol.
                    \begin{enumerate}
                        \item
                            Generate two large prime number $p$ and $q$.
                        \item
                            Define $n = pq$.
                        \item
                            Choose some $e \in [n]$ such that $e$ and $\eulerphi{n}$ are
                            relatively prime.
                        \item
                            Find $d$ such that $ed \equiv 1 \mod \eulerphi{n}$.
                        \item 
                            Now Bob's private key is the triple $(p, q, d)$ and his public
                            key is $(n, e)$. Bob must keep his private key secret, but 
                            can publish his private key to the rest of the world.
                    \end{enumerate}
                    Now Bob is ready to recieve messages from anyone else in the world.
                \item
                    Now if Alice wants to securely send Bob a message she simply follows 
                    the following steps.
                    \begin{enumerate}
                        \item
                            Alice first retrieves Bob's public key: $(n, e)$ as defined above.
                        \item
                            She then takes her message and breaks it up into blocks of size $\ceil{\lg n}$.
                        \item
                            Now Alice replaces each block, $m$, of her message with
                            the cypher text generated by the following encryption function:
                            \begin{equation}
                                E_{n, e}(m) = m^e \mod n
                            \end{equation}
                        \item
                            Alice can now send the cipher text of her message to Bob. Note Alice's
                            message now looks like
                            \[
                                E_{n, e}(m_1) \ | \ E_{n, e}(m_2) \ | \ \cdots \ | \ E_{n, e}(m_k)
                            \]
                            where $k = \ceil{\frac{\abs{m}}{\ceil{\lg n}}}$ is the number of message blocks
                            and $|$ denote concatenation.
                    \end{enumerate}
                    Using this method Alice can send any message she wants to Bob.
                \item
                    Once Bob recieves Alice's message over some sort of network connection
                    he can decrypt her message by following these steps.
                    \begin{enumerate}
                        \item
                            Bob first retrieves his private key: $(p, q, d)$ and his public key: $(n, e)$
                            as defined above.
                        \item   
                            He then takes the cipher text and breaks it up into blocks of size $\ceil{\lg n}$. 
                        \item
                            Now Bob decodes each block, $c$, of the cipher text using the following
                            function:
                            \[
                                D_{n, d}(c) = c^d \mod n
                            \]
                        \item
                            Bob should now Bob has the message Alice sent in the form
                            \[
                                D_{n, d}(c_1) \ | \ D_{n, d}(c_2) \ | \ \cdots \ | \ D_{n, d}(c_k)
                            \]
                            where $k = \ceil{\frac{\abs{m}}{\ceil{\lg n}}}$ is the number of message blocks
                            and $|$ denote concatenation.
                    \end{enumerate}
                    Bob has now successfully recieved Alice's message.
            \end{itemize}
            \begin{claim}
                $``$If Alice and Bob follow the protocol outlined above then Bob
                will always successfully recieve Alice's message.''
            \end{claim}
            \begin{proof}
                Assume that Bob want to recieve a message from Alice, so he generates a
                public-private key pair $(n, e)$ and $(p, q, d)$ such that
                \begin{align}
                    p, q & \in \set{P} \\
                    n & = pq \\
                    \gcd{e}{\eulerphi{n}} & = 1 \\
                    \label{ed requirement}
                    ed & \equiv 1 \mod \eulerphi{n}
                \end{align}
                Now assume that Alice sends Bob a message, $m$, using the RSA protocol. To show
                that Bob will always recieve her message unaltered it suffices to show that
                Bob will recieve any block of size $k = \ceil{\frac{\abs{m}}{\ceil{\lg n}}}$.

                Let $i \in [k]$ be given. We now want to show that block $m_i$ of Alices message
                will always be recieved by Bob. If Alice is following the protocol then she will send
                \begin{equation}
                    c_i = E_{n, e}(m_i) = m_i^e \mod n
                \end{equation}
                to Bob. Now when Bob recieves the cipher text block $c_i$ from Alice, then if
                he is following the protocol he will decrypt it as follows:
                \begin{equation}
                    \label{rsa decrypt}
                    D_{n, d}(c_i) = c_i^d \mod n
                \end{equation}
                We now note that since $c_i$ and $m_i^e$ are congruent modulo $n$ that we 
                can substitute $m_i^e$ for $c_i$ is equation ($\ref{rsa decrypt}$).
                \begin{equation}
                    \label{rsa decrypt 2}
                    D_{n, d}(c_i) = (m_i^e)^d \mod n = m_i^{ed} \mod n
                \end{equation}
                We will now proceed to show that $D_{n, d}(c_i)$ is always equal to $m_i$
                in two parts. We will first show that $D_{n, d}(c_i) \equiv m_i \mod p$ and
                $D_{n, d}(c_i) \equiv m_i \mod q$. Then we will show that if any $x \equiv m_i \mod p$
                and $x \equiv m_i \mod q$, then $x \equiv m_i \mod pq$. The result then follows
                from Modus Ponens.
                \begin{itemize}
                    \item
                        Let $P \in \set{P}$ be an aribtrary prime number. We now show that
                        $D_{n, d}(c_i) \equiv m_i \mod P$. There are two cases: either $P$ is
                        a divisor of $m_i$ or not.
                        \begin{itemize}
                            \item
                                Assume $P$ is a divisor of $m_i$.
                                \begin{derivation}{\equiv}
                                    D_{n, d}(c_i) & m_i^{ed} \mod P & \just{By equation $\ref{rsa decrypt 2}$.}
                                                  & 0^{ed} \mod P & \just{Since $\divides{P}{m_i}$.}
                                                  & 0 \mod P & \\
                                                  & m \mod P & \just{Since $\divides{P}{m_i}$.}
                                \end{derivation}
                            \item
                                Assume $P$ is not a divisor of $m_i$. This implies that $P$
                                is relatively prime to $m_i$ since $P$ is prime.
                                \begin{derivation}{\equiv}
                                    D_{n, d}(c_i) & m_i^{eq} \mod P \\
                                                  & m_i^{k \eulerphi{n} + 1} \mod P & \just{By equation $\ref{ed requirement}$.}
                                                  & m_i \cdot m_i^{k(p - 1)(q - 1)} \mod P & \just{Since $\eulerphi{n} = (p - 1)(q - 1)$.}
                                                  & m_i \cdot \left(m_i^{p - 1}\right)^{k(q - 1)} \mod P \\
                                                  & m_i \cdot 1^{k(q - 1) + 1} \mod P & \just{By Theorem $\ref{Fermat's Little Theorem}$.}
                                                  & m_i \cdot 1 \mod P \\
                                                  & m_i \mod P 
                                \end{derivation}
                                Thus for any prime $P$ we know that $D_{n, d}(c_i) = m_i$, namely:
                                \begin{align}
                                    D_{n, d}(c_i) = m^{ed} \equiv m \mod p \\
                                    D_{n, d}(c_i) = m^{ed} \equiv m \mod q
                                \end{align}
                        \end{itemize}
                    \item
                        We now use the Chinese Remainder Theorem to show that for any $u \in \set{Z}$,
                        then if $u \equiv m_i \mod p$ and $u \equiv m_i \mod q$, then
                        $u \equiv m_i \mod pq$. So let $u \in \set{Z}$ be given such that
                        \begin{align*}
                            u &\equiv m_i \mod p \\
                            u &\equiv m_i \mod q \\
                        \end{align*}
                        Since $p$ and $q$ are trivially relatively prime we know by Theorem
                        \ref{Chinese Remainder Theorem} That there is a unique $u$ modulo
                        $pq$ that satisfies our system of congruences. Furthermore, the proof
                        of the Chinese Remainder Theorem tells us that this $u$ is congruent to
                        \[
                            m_ipx + m_iqy
                        \]
                        where $x$ and $y$ are integers such that $px + qy = 1$. Thus we can
                        rewrite our solution as follows
                        \[
                            m_ipx + m_iqy = m_i(px + qy) = m_i
                        \]
                        Now we can finally conclude that $u \equiv m_i \mod pq$ as required. \QED
                \end{itemize}
                Thus by Modus Ponens we can conclude that $D_{n, d}(c_i) = m_i$, i.e. Bob successfully
                received Alice's message. \QED
            \end{proof}
        \end{definition}
    \section{Common Divisibility Rules}
        \begin{theorem}
            ``An integer is divisible by $3$ if and only if the sum of the digits in its base 10
            representation is divisible by $3$.''
        \end{theorem}
        \begin{proof}
            Let $n \in \set{Z}$ be given. Suppose the base 10 representation of $n$ is
            $d_k \cdots d_1 d_0$, which can be written as
            \[
                n = \sum_{i = 0}^{k} d_i \cdot 10^i
            \]
            We now simply note that $10^i \equiv 1 \mod 3$ for all $i \in \set{N}$. We now simply
            analyze the base 10 representation of $n$ modulo 3 as follows
            \[
                n \equiv \sum_{i = 0}^{k} d_i \cdot 10^i \equiv \sum_{i = 0}^{k} d_i \cdot 1 \equiv \sum_{i = 0}^{k} d_i \mod 3
            \]
            Thus we have shown the $n$ and the sum of the digits in the base 10 representation of
            $n$ are in the same equivalence class modulo 3. Thus we can conclude that $n$
            is divisible by 3 if and only if the sum of the digits in the base 10 representation of $n$
            is divisible by 3. \QED
        \end{proof}
        \begin{theorem}
            $``$Let $b, n \in \set{N}$ and let $n = \sum_{i = 0}^{k} n_i \cdot b^i$ be the base
            $b$ representation of $n$. Then $b + 1$ divides $n$ if and only if $b + 1$ divides
            $\sum_{i = 0}^{k} (-1)^{i} \cdot n_i$.''
        \end{theorem}
        \begin{proof}
            Let $b, n \in \set{N}$ be given. Consider the base $b$ represenation of $n$
            \[
                n = \sum_{i = 0}^{k} n_i \cdot b^i
            \]
            We first note that $\divides{(b + 1)}{n}$ if and only if $n \equiv 0 \mod (b + 1)$.
            We also note that $b \equiv -1 \mod (b + 1)$. Thus we can now make the following
            simple calculation:
            \[
                n \equiv \sum_{i = 0}^{k} n_i \cdot b^i \equiv  \sum_{i = 0}^{k} b_i \cdot (-1)^i \mod (b + 1)
            \]
            Since congruence modulo $b + 1$ is an equivalence relation we can conclude that
            $n \equiv 0 \mod (b + 1)$ if and only if $\dsum_{i = 0}^{k} b_i \cdot (-1)^i \equiv 0 \mod (b + 1)$.
            But this just means that $(b + 1)$ divides ${\dsum_{i = 0}^{k} b_i \cdot (-1)^i}$. \QED
        \end{proof}
        \begin{corollary}
            $``$An integer is divisible by 11 if and only if the alternating sum of the digits
            in its base 10 representation is divisible by 11.''
        \end{corollary}
        \begin{theorem}
            ``$\forall a, b, n \in \setsub{\set{N}}{\Set{1}} \ \ndivides{(a^n - b^n)}{(a^n + b^n)}$.'' 
        \end{theorem}
        \begin{proof}
            % TODO
        \end{proof}
        \begin{theorem}
            ``$\forall n, s \in \set{N} \ \divides{n!}{\prod_{k = 0}^{n - 1} (s + k)}$.''
        \end{theorem}
        \begin{proof}
            % TODO
        \end{proof}
        % TODO
    \section{Continued Fractions}
        \begin{definition}
            Let $r$ be a real number. We define the continued fractions representation of $r$ as
            \[
                r = \frac{q_0}{q_1 +
                    \frac{1}{q_2 + 
                        \frac{1}{q_3 + 
                            \frac{1}{\ddots}
                        }
                    }
                }
            \]
            such that $\forall i \in \set{N} \ q_i \in \set{Z}$. For simply we write that
            $r = [q_0; q_1, q_2, \dots]$ to denote the above continued fractions representation
            of $r$. We also call the $q_i$'s partial quotients of $[q_0; q_1, q_2, \dots]$.
        \end{definition}
        \begin{theorem}
            ``The continued fractions representation of any rational number has a finite number
              of partial quotients.''
        \end{theorem}
        \begin{proof}
            % TODO
        \end{proof}
        \begin{definition}
            % Define convergents.
        \end{definition}
        \begin{theorem}
            % Fast left to right generation of convergents.
        \end{theorem}
    \section{Unique Base $q$ Representation}
        \begin{definition}
            Let $q \in \set{N}$ and $n \in \set{Z}$. We define $d_k\cdots d_1$ to be a base
            $q$ representation of $n$ if and only if the following hold
            \begin{itemize}
                \item
                    $d_k \neq 0$
                \item
                    $\forall i \in [k] \ d_i \in \unionzero{[q - 1]}$
                \item
                    $n = \dsum_{i = 1}^k d_i \cdot q^{i - 1}$
            \end{itemize}
            We represent negative integers by prepending a $-$ to the representation. We also
            say $d_k\cdots d_1$ has length $k$. We will write $n = \base{d_k \cdots d_1}{q}$ to indicate that
            $d_k \cdots d_1$ is a base $q$ representation of $n$.
        \end{definition}
        \begin{theorem}
            $``\forall q \in \set{N} \ \forall n \in \set{Z} \ n$ has a unique base $q$ represenation.''
        \end{theorem}
        \begin{proof}
            Let $n \in \set{Z}$ and $q \in \set{N}$ be given. If $n = 0$ we know that
            $n = \base{0}{q}$, so assume $n \neq 0$. Furthermore we note that if $n < 0$,
            then the base $q$ representation of $n$ is just the base $q$ representation of $|n|$
            with a $-$ prepended to it. Thus we can assume $n > 0$. We can now proceed to show
            the existence and uniqueness of base $q$ representation of $n$.
            \begin{itemize}
                \item
                \item
            \end{itemize}
            Thus we can conclude that $n$ has a unique base $q$ representation. \QED
        \end{proof}
    \section{$\sqrt{2}$ is Irrational}
        \begin{lemma}
            $``\forall a \in \set{Z} \ a^2 \equiv a \mod 2.$''
        \end{lemma}
        \begin{proof}
            Let $a \in \set{Z}$ be given. We now have two cases based on the parity of $a$.
            \begin{itemize}
                \item
                    If $a$ is even we can see that $a \equiv 0 \mod 2$. This implies that
                    $a^2 \equiv 0^2 \equiv 0 \mod 2$. Thus we can see that $a^2 \equiv a \mod 2$.
                \item
                    If $a$ is odd we can see that $a \equiv 1 \mod 2$. This implies that
                    $a^2 \equiv 1^2 \equiv 1 \mod 2$. Thus we can see that $a^2 \equiv a \mod 2$.
            \end{itemize}
            Thus we can see that $a^2 \equiv a \mod 2.$. \QED
        \end{proof}
        \begin{theorem}
            $``\sqrt{2} \not\in \set{Q}.$''
        \end{theorem}
        \begin{proof}
            We proceed via a proof by contradiction, thus assume that $\sqrt{2} \in \set{Q}$,
            i.e. assume that $\sqrt{2}$ is rational. This implies that there are integers
            $a$ and $b$ such that $\sqrt{2} = \dfrac{a}{b}$. Furthermore we will insist that
            $\gcd{a}{b} = 1$, i.e. we want $\dfrac{a}{b}$ to be in lowest terms. Now we can square
            both sides of the equation and deduce 
            \[
                \sqrt{2}^2 = \left(\dfrac{a}{b}\right)^2 \then 2 = \dfrac{a^2}{b^2} \then a^2 = 2b^2
            \]
            Since we now know that $a^2 = 2b^2$ we can conclude that $a^2$ must be even. By the
            previous Lemma we can conclude that $a$ must also be even. Thus we know there is an
            integer $c$ such that $a = 2c$. Substituting the new form of $a$ we can conclude
            \[
                a^2 = 2b^2 \then (2c)^2 = 2b^2 \then 4c^2 = 2b^2 \then b^2 = 2c^2
            \]
            Since we know that $b^2 = 2c^2$ we can conclude that $b^2$ must also be even. Again
            by the previous Lemma we can conclude that $b$ must also be even. This of course implies
            that the $\gcd{a}{b} = 2 \neq 1$, thus $\dfrac{a}{b}$ was not a lowest terms representation
            of $\sqrt{2}$. This is of course a contradiction. \QED
        \end{proof}
    \section{$e$ is Irrational}
        \begin{definition}
            We define $e$ and $S_n$, for $n \ge 0$, as
            \begin{align*}
                e = \dsum_{k = 0}^{\infty} \frac{1}{k!}
                & & S_n = \dsum_{k = 0}^{n} \frac{1}{k!}
            \end{align*}
        \end{definition}
        \begin{lemma}
            \label{e is Irrational Lemma}
            $``\forall n \in \set{N} \ \ 0 < n! (e - S_n) < \dfrac{1}{n}$.''
        \end{lemma}
        \begin{proof}
            Let $n \in \set{N}$ be given. We first show that $e - S_n > 0$.
            \begin{align*}
                e - S_n &= \sum_{k = 0}^{\infty} \frac{1}{k!} - \sum_{k = 0}^{n} \frac{1}{k!}
                        & \just{By definition of $e$ and $S_n$.}
                        &= \sum_{k = n + 1}^{\infty} \frac{1}{k!} \\
            \end{align*}
            Since each term of the sum above is positive we can conclude that $e - S_n > 0$.
            We now show that $e - S_n < \frac{1}{n}$ by continuing the computation above.
            \begin{align*}
                e - S_n &= \sum_{k = n + 1}^{\infty} \frac{1}{k!} & \just{As above}
                        &= \frac{1}{(n + 1)!} \sum_{k = n + 1}^{\infty} \frac{(n + 1)!}{k!}
                        & \just{By factoring out $(n + 1)!$.}
            \end{align*}
            We now note that for $k \ge n + 1$
            \begin{equation}
                \label{sum term equation}
                \frac{(n + 1)!}{k!} = \left(\prod_{j = n + 2}^{k} j\right)^{-1}
            \end{equation}
            since we can cancel out a factor of $(n + 1)!$ from both the numerator and
            denominator. We now want to upper bound equation
            (\ref{sum term equation}). We do this by making each term of the product
            smaller; thus making the reciprocal of the product larger.
            \begin{align*}
                \frac{(n + 1)!}{k!} &= \left(\prod_{j = n + 2}^{k} j \right)^{-1}
                                    & \just{By equation (\ref{sum term equation}).}
                                    &< \left(\prod_{j = n + 2}^{k} (n + 1) \right)^{-1}\\
            \end{align*}
            Thus we can now conclude that
            \begin{equation}
                \label{sum term equation 2}
                \frac{(n + 1)!}{k!} < \frac{1}{(n + 1)^{k - (n + 1)}}
            \end{equation}
            We can now continue our computation of $e - S_n$.
            \begin{align*}
                e - S_n &= \frac{1}{(n + 1)!} \sum_{k = n + 1}^{\infty} \frac{(n + 1)!}{k!} \\
                        &< \frac{1}{(n + 1)!} \sum_{k = n + 1}^{\infty} \frac{1}{(n + 1)^{k - (n + 1)}}
                        & \just{By equation $(\ref{sum term equation 2})$.}
                        &= \frac{1}{(n + 1)!} \sum_{k = 0}^{\infty} \frac{1}{(n + 1)^{k}}
                        & \just{By re-indexing the sum.}
                        &= \frac{1}{(n + 1)!} \frac{1}{1 - \frac{1}{n + 1}}
                        & \just{Since the sum is a geometric series.}
                        &= \frac{1}{(n + 1)!} \frac{n + 1}{n} \\
                        &= \frac{1}{n! \cdot n}
            \end{align*}
            Thus we can conclude that $0 < e - S_n < \frac{1}{n! \cdot n}$. We can now multiply
            this inequality by $n!$ to conclude that
            \begin{equation}
                0 < n! (e - S_n) < \frac{1}{n}
            \end{equation}
            as required. \QED
        \end{proof}
        \begin{theorem}
            $``e \not\in \set{Q}$.''
        \end{theorem}
        \begin{proof}
            We proceed via contradiction, so assume that $e \in \set{Q}$. In other
            words assume there are $a, b \in \set{N}$ such that $b \neq 0$ and $e = \dfrac{a}{b}$.
            Now choose $n \in \set{N}$ such that $n > b$, in particular note that $n > b \ge 1$.
            We can now invoke Lemma \ref{e is Irrational Lemma} to conclude that
            \begin{equation*}
                0 < n! (e - S_n) < \frac{1}{n}
            \end{equation*}
            Since $e = \dfrac{a}{b}$ we can conclude that
            \begin{equation*}
                0 < n! \left(\frac{a}{b} - S_n\right) < \frac{1}{n}
            \end{equation*}
            We can rearrange this formula and use the fact that $\frac{1}{n} < 1$ to conclude that
            \begin{equation}
                0 < a \left(\frac{n!}{b}\right) - S_n n! < \frac{1}{n} < 1
            \end{equation}
            In particular the inequality above tells us that $a \left(\frac{n!}{b}\right) - S_n n!$
            cannot be an integer.

            We will now show that this above quantity must be an integer thus giving us our contradiction.
            First note that $b$ is a factor of $n!$ since $n > b$.
            This implies that $\frac{n!}{b}$ is an integer. Since $a$ is an integer and the
            integers are closed under multiplication we can conclude that $a \left(\frac{n!}{b}\right)$
            is also an integer. Now lets expand out $S_n n!$
            \begin{align*}
                S_n n! &= n! \sum_{k = 0}^{n} \frac{1}{k!} & \just{By definition of $S_n$.}
                       &= \sum_{k = 0}^{n} \frac{n!}{k!} 
            \end{align*}
            We first note that each of terms of the above summation must be an integer since
            for each $k \in \Set{0, 1, \dots, n}$ we know that $k!$ is a factor of $n!$. Since
            the integers are closed under addition we can conclude that the entire sum is an
            integer, i.e. $S_n n!$ must be an integer. Now since the integers are closed under
            subtraction we can conclude that $a\left(\frac{n!}{b}\right) - S_n n!$ must be an
            integer; which is of course a contradiction. Thus we can conclude that $e \not\in \set{Q}$
            as required. \QED
        \end{proof}
    \section{$\pi$ is Irrational}
        \begin{definition}
            % TODO
        \end{definition}
        \begin{theorem}
            % TODO
        \end{theorem}
        \begin{proof}
            % TODO
        \end{proof}
    \section{Harmonic Numbers}
        \begin{definition}
            Let $n \in \set{N}$. We define the $n^{th}$ Harmonic Number, denoted $H_n$, as follows
            \[
                H_n = \sum_{i = 1}^{n} \frac{1}{i} = 1 + \frac{1}{2} + \frac{1}{3} + \frac{1}{4} + \dots
            \]
        \end{definition}
        \begin{theorem}
            $``\forall n \in \set{N} \ H_{2^n} > n.$''
        \end{theorem}
        \begin{proof}
            We proceed by induction on $n$ to prove $\inductive{P}{n}$ for each $n \in \set{N}$.
            \[
                \inductive{P}{n} = ``H_{2^n} > n."
            \]
            \begin{itemize}
                \item
                    Base Case: Let $n = 1$, then $H_{2^1} = H_2 = 1 + \frac{1}{2} = \frac{3}{2} > 1$
                    as required.
                \item
                    Induction Step: Let $k \in \set{N}$ be given, assume $\inductive{P}{k}$ is true.
                    We will now simply try to bound $H_{2^{k + 1}}$ by $k + 1$.
                    \begin{derivation}{=}
                        H_{2^{k + 1}} & \sum_{i = 1}^{2^{k + 1}} \frac{1}{i} 
                                      & \just{By definition of $H_{2^{k + 1}}$.}
                                      & \left(\sum_{i = 1}^{2^{k}} \frac{1}{i}\right) + 
                                        \left(\sum_{i = 2^k + 1}^{2^{k + 1}} \frac{1}{i}\right) \\
                                      & H_{2^k} + \left(\sum_{i = 2^k + 1}^{2^{k + 1}} \frac{1}{i}\right) 
                                      & \just{By definition of $H_{2^k}$.}
                        \multicolumn{1}{l@{\ > \ }}{}
                                      & k + \left(\sum_{i = 2^k + 1}^{2^{k + 1}} \frac{1}{i}\right) 
                                      & \just{By $\inductive{P}{k}$.}
                        \multicolumn{1}{l@{\ > \ }}{}
                                      & k + \left(\sum_{i = 2^k + 1}^{2^{k + 1}} \frac{1}{2^k}\right)
                                      & \just{By $\frac{1}{2^k} > \frac{1}{j}$ 
                                              if $2^{k} + 1 \le j \le 2^{k + 1}$.}
                                      & k + 2^{k} \left(\frac{1}{2^k}\right) \\
                                      & k + 1
                    \end{derivation}
                    Thus $H_{2^{k + 1}} > k + 1$ as required.
            \end{itemize}
            Thus we can conclude that $\inductive{P}{n}$ is true for all $n \in \set{N}$. \QED
        \end{proof}
        \begin{corollary}
            $``H_{\infty} = \dlim_{n \rightarrow \infty} H_n$ diverges.''
        \end{corollary}
    \section{The AKS Primality Test}
        % TODO Give the algorithm and prove its correctness/runtime.

