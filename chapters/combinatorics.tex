% name: Sam Tetruashvili
% title: combinatorics.tex
% date created: Fri Dec  5 16:07:32 EST 2008
% description: The Combinatorics chapter of The Book of Proofs.

% last modified: Fri Aug  6 21:32:48 PDT 2010

\chapter{Combinatorics}
    \section{The Pigeonhole Principle}
        \begin{theorem}
        \end{theorem}
        \begin{proof}
            % TODO
        \end{proof}
    \section{Binomial Coefficients}
        \begin{definition}
            Let $n \in \set{N}$ and $k \in \set{Z}$ such that $0 \le k \le n$.
            We define $\binom{n}{k}$, read $n$ choose $k$, to be the number of
            $k$ element subsets of an $n$ element set. Also note that $\binom{n}{k}$
            is called a binomial coefficient.
        \end{definition}
        \begin{lemma}
            $``\forall n \in \set{N} \ \forall k \in \union{[n]}{\Set{0}} \
            \binom{n}{k} = \binom{n - 1}{k} + \binom{n - 1}{k - 1}.$''
            \label{rec binom lemma}
        \end{lemma}
        \begin{proof}
            Let $n \in \set{N}$ and $k \in \union{[n]}{\Set{0}}$ be given. We now proceed
            by counting the set, $S$, of $k$ element subsets of $[n]$ in two ways.
            \[
                S = \Set{\subs{A}{[n]} \mid |A| = k}
            \]
            \begin{itemize}
                \item
                    We know that by the definition of $\binom{n}{k}$ that $S$ has $\binom{n}{k}$
                    elements.
                \item
                    We will now try to count the number of elements of $S$ by partitioning it
                    based on whether or not 1 is contained in a subset. Thus we define the
                    partitions $S_1$ and $S_2$ as follows:
                    \[
                        S_1 = \Set{A \in S \mid 1 \in A} \mbox{ and } S_2 = \Set{A \in S \mid 1 \not\in A}
                    \]
                    We know that $S_1$ has $\binom{n - 1}{k - 1}$ elements because
                    any $A \in S_1$ must contain 1, thus we need to choose
                    $k - 1$ of the remaining $n - 1$ elements of $[n]$ to be in $A$. We also
                    know that $S_2$ has $\binom{n - 1}{k}$ elements because any $A \in S_2$ must
                    no contain 1, thus we need to choose $k$ of the remaining $n - 1$ of $[n]$ to
                    be in $A$. Since we know $S_1$ and $S_2$ partition $S$ we can now conclude
                    via the rule of sum that $|S| = |S_1| + |S_2| = \binom{n - 1}{k - 1} + \binom{n - 1}{k}.$
            \end{itemize}
            Thus we can conclude that $\binom{n}{k} = \binom{n - 1}{k} + \binom{n - 1}{k - 1}$. \QED
        \end{proof}
        \begin{theorem}[Binomial Theorem]
            $``\forall n \in \set{N} \ \forall a, b \in \set{R} \ (a + b)^2 = \dsum_{i = 0}^{n} \binom{n}{i}a^ib^{n - i}.$''
            \label{Binomial Theorem}
        \end{theorem}
        \begin{proof}
            Let $n \in \set{N}$ and $a, b \in \set{R}$ be given. Consider the following:
            \[
                (a + b)^n = \underbrace{(a + b)(a + b) \cdots (a + b)}_{\mbox{$n$ terms}}
            \]
            To construct a term in the closed form of $(a + b)^n$ we must first choose the power
            of $a$'s in the term, this value ranges from 0 to $n$. Assume we choose to have
            $i \in \union{[n]}{\Set{0}}$ $a$'s; this
            implies that we have $n - i$ $b$'s in the term. So now we just have to go through each
            of the $n$ binomial terms and choose if we will take an $a$ or $b$ from it. Thus we
            are going through the $n$ terms and choosing $i$ of them to take an $a$ from; so there
            are $\binom{n}{i}$ different ways to choose $i$ $a$'s from the $n$ terms. Thus we can
            conclude that $(a + b)^n = \sum_{i = 0}^{n} \binom{n}{i} a^ib^{n - i}$. \QED
        \end{proof}
        \begin{proof}[Alternate Proof]
            We may also proceed by induction on $n$ to prove $\inductive{P}{n}$ for all
            $n \in \unionzero{\set{N}}$.
            \[
                \inductive{P}{n} = ``\forall x, y \in \set{R} \ (x + y)^n = \sum_{k = 0}^{n} \binom{n}{k} x^k y^{n - k}."
            \]
            \begin{itemize}
                \item
                    Base Case: Let $x, y \in \set{R}$ be given. We can clearly see that
                    \[
                        (x + y)^0 = 1 = \binom{0}{0} x^0 y^{0 - 0}
                                  = \sum_{k = 0}^{0} \binom{0}{k} x^k y^{n - K}
                    \]
                    as required. Thus $\inductive{P}{0}$ is true.
                \item
                    Induction Step: Let $n \in \unionzero{\set{N}}$ and $x, y \in \set{R}$
                    be given; assume $\inductive{P}{n}$ is true. We will now simply
                    compute $(x + y)^{n + 1}$ to establish that $\inductive{P}{n + 1}$ is
                    true.
                    \[
                        (x + y)^{n + 1} = (x + y)(x + y)^n
                    \]
                    By the inductive hypothesis we can know that
                    $(x + y)^n = \sum_{k = 0}^n \binom{n}{k} x^k y^{n - k}$. Thus
                    substituting this gives us the following.
                    \begin{derivation}{=}
                        (x + y)^{n + 1} & (x + y)\dsum_{k = 0}^n \dbinom{n}{k} x^k y^{n - k} \\
                                        & \dsum_{k = 0}^n \dbinom{n}{k} x^{k + 1} y^{n - k} +
                                          \dsum_{k = 0}^n \dbinom{n}{k} x^k y^{n - k + 1}
                    \end{derivation}
                    We now take the last term out of the first summation and first term
                    out of the second summation.
                    \[
                        \dsum_{k = 0}^{n - 1} \dbinom{n}{k} x^{k + 1} y^{n - k} + \dbinom{n}{n} x^{n + 1}y^0 +
                        \dbinom{n}{0} x^0y^{n + 1} + \dsum_{k = 1}^{n} \dbinom{n}{k}x^k y^{n - k + 1}
                    \]
                    We now note that $\binom{n}{n} = \binom{n}{0} = \binom{n + 1}{n + 1} = \binom{n + 1}{0} = 1$
                    and then re-index the first summation as follows.
                    \[
                        \dsum_{k = 1}^{n} \dbinom{n}{k - 1} x^{k} y^{n - (k + 1)} + \dbinom{n + 1}{n + 1} x^{n + 1}y^0 +
                        \dbinom{n + 1}{0} x^0y^{n + 1} + \dsum_{k = 1}^{n} \dbinom{n}{k}x^k y^{n - k + 1}
                    \]
                    We can now combine the two summations
                    \[
                        \binom{n + 1}{0} x^0y^{n + 1} +
                        \sum_{k = 1}^{n} \left(\binom{n}{k} + \binom{n}{k - 1}\right) x^{k} y^{(n + 1) - k)} +
                        \binom{n + 1}{n + 1} x^{n + 1}y^0
                    \]
                    We can now invoke \LemmaRef{rec binom lemma} to simply $\binom{n}{k} + \binom{n}{k - 1}$
                    to $\binom{n + 1}{k}$.
                    \[
                        \binom{n + 1}{0} x^0y^{n + 1} +
                        \sum_{k = 1}^{n} \binom{n + 1}{k}x^{k} y^{(n + 1) - k)} +
                        \binom{n + 1}{n + 1} x^{n + 1}y^0
                    \]
                    We now bring the two non-summation terms into the summation and conclude
                    that
                    \[
                        (x + y)^{n + 1} = \sum_{k = 0}^{n + 1} \binom{n + 1}{k} x^k y^{(n + 1) - k}
                    \]
                    Thus $\inductive{P}{k + 1}$ is true as required.
            \end{itemize}
            Thus we can conclude that $\inductive{P}{n}$ is true for all $n \in \unionzero{\set{N}}$
            as required. \QED
        \end{proof}
        \begin{theorem}
            $``$Let $n$ and $k$ be positive integers. There are $\binom{n - 1}{k - 1}$ distinct
            solutions to the equation $x_1 + x_2 + \cdots + x_k = n$ when the $x_i$'s are restricted
            to being positive integers.''
            \label{Pirates and Gold 1}
        \end{theorem}
        \begin{proof}
            Let $n$ and $k$ be positive integers. Now interpret the equation
            \begin{equation}
                \label{Pirates and Gold Eqation}
                x_1 + x_2 + \cdots + x_k = n
            \end{equation}
            as placing $n$ indistinguishable balls into $k$ distinguishable bins
            (bins 1, 2, \dots, $k$). Now arbitrarily number the balls from $1$ to $n$ and
            then place them in order on a line. Place a barrier between any two
            consecutive balls; also place a barrier to the left of the first ball and
            to the right of the last ball. We now number these barriers from $0$ to $n$, where
            the barrier to the right of ball $i$ is barrier $i$ and barrier 0 is the
            one to the left of ball $0$. This can be visualized in \FigureRef{Balls in a Row}.
            \begin{figure}[H]
                \centering
                \[
                    \mid_{0} \ball_1 \mid_1 \ball_2 \mid_2 \cdots \mid_{n - 1} \ball_n \mid_{n}
                \]
                \caption{This image represents the $n$ balls and the $n + 1$ barriers. Ball $i$
                         is represented as $\ball_i$ and barrier $i$ is represented as $\mid_i$.}
                \label{Balls in a Row}
            \end{figure}
            Now to generate an assignment to the $x_i$'s we're simply going to pick $k - 1$
            of the $n - 1$ eligible barriers; say the indices of these barriers are
            $b_1, b_2, \dots, b_{k - 1}$. We will put all of the balls between barrier
            $b_{i - 1}$ and barrier $b_i$ into bin $i$. Barriers 0 and $n$ are not eligible to be
            chosen because if we pick these barriers we will lose a region and we need to define
            $k$ regions as there are $k$ bins. For convenience sake we will define $b_0 = 0$ and
            $b_k = n$. Now we can define our $x_i$'s (the number of balls in bin $i$) in terms
            of our $b_i$'s as follows:
            \begin{equation}
                x_i = b_i - b_{i - 1} 
            \end{equation}
            Each of the $x_i$'s is always positive as required. Every valid solution can be
            generated in this way. So all we have to do now is count how many ways we can pick
            $b_1, b_2, \dots, b_{k - 1}$. Well we have $n - 1$ possible barriers to pick from
            (barriers 1 through $n - 1$) and we have to pick $k - 1$ of them to define $k$ regions,
            i.e. we are picking $k - 1$ elements from a $n - 1$ element set. Thus there are exactly
            $\binom{n - 1}{k - 1}$ ways to pick the barriers, and thus exactly $\binom{n - 1}{k - 1}$
            distinct solutions to \equationRef{Pirates and Gold Eqation} as required. \QED
        \end{proof}
        \begin{corollary}[Pirates and Gold]
            $``$Let $n$ and $k$ be positive integers. There are $\binom{n + k - 1}{k - 1}$ distinct
            solutions to the equation $x_1 + x_2 + \cdots + x_k = n$ when the $x_i$'s are restricted
            to being non-negative integers.''
            \label{Pirates and Gold}
        \end{corollary}
        \begin{note}
            \CorollaryRef{Pirates and Gold} is called Pirates and Gold because it represents
            the number of ways to distribute $n$ indistinguishable bars of gold to $k$
            distinguishable pirates.
        \end{note}
        \begin{proof}
            The only way in which this is different from \TheoremRef{Pirates and Gold 1}
            is that the $x_i$'s are now allowed to equal 0. To get around this we can
            guarantee each of the bins at least 1 ball and then remove 1 ball from each bin
            after we've distributed the balls to the bins. In particular define $y_i$ for
            $i = 1, 2, \dots, k$ as follows
            \begin{equation}
                y_i = x_i + 1
            \end{equation}
            Each solution the equation
            \begin{equation}
                x_1 + x_2 + \cdots + x_k = n
                \label{Pirates and Gold equation 2}
            \end{equation}
            is now mapped to a solution to the equation
            \begin{equation}
                y_1 + y_2 + \cdots + y_k = n + k
                \label{Pirates and Gold equation 1}
            \end{equation}
            Since the $x_i \ge 0$ we know that $y_i = x_i + 1 \ge 1$. Thus we can use
            \TheoremRef{Pirates and Gold 1} to conclude that there are exactly $\binom{n + k - 1}{k - 1}$
            distinct solutions to \equationRef{Pirates and Gold equation 1}. Since each solution to
            \equationRef{Pirates and Gold equation 1} can be mapped to a solution to
            \equationRef{Pirates and Gold equation 2}, by taking
            \begin{equation}
                x_i = y_i - 1
            \end{equation}
            we can conclude that there are $\binom{n + k - 1}{k - 1}$ distinct solutions
            to \equationRef{Pirates and Gold equation 2} when the $x_i$'s all non-negative
            integers. \QED
        \end{proof}
        \begin{theorem}[Chairperson Identity]
            $``\forall n \in \set{N} \ \forall k \in [n] \ k \binom{n}{k} = n \binom{n - 1}{k - 1}$.''
        \end{theorem}
        \begin{proof}
            Let $n \in \set{N}$ and $k \in [n]$ be given. Let $S$ be defined as follows
            \begin{equation}
                S = \Set{(c, C) \ \mid \ \subs{C}{[n]}, |C| = k \mbox{, and } c \in C}
            \end{equation}
            We will refer to $C$ as the committee and $c$ as the chairperson.
            We now show that $k \binom{n}{k} = n \binom{n - 1}{k - 1}$ by counting $S$ in
            two ways.
            \begin{itemize}
                \item
                    We first count $S$ by picking the committee and then out of the people
                    in the committee picking the chairperson. So to make an element of $S$
                    we can follow this simple two step process:
                    \begin{enumerate}
                        \item
                            Pick $C$ to be $k$ element subset of $[n]$ to be the committee.
                        \item
                            Pick $c \in C$ to be the chairperson.
                    \end{enumerate}
                    There are $\binom{n}{k}$ to do the first step. Since $C$ is a $k$
                    element set there are $k$ ways to do the second step. Thus by the
                    rule of product we can conclude that $|S| = k \binom{n}{k}$.
                \item
                    We now count $S$ by first picking the chairperson and then from the
                    remaining $n - 1$ people picking $k - 1$ to fill out the committee.
                    We can make an element of $S$ by following this simple two step process:
                    \begin{enumerate}
                        \item
                            Pick $c \in [n]$ to be the chairperson.
                        \item
                            Pick $\subs{A}{\setsub{[n]}{\Set{c}}}$ and let $C = \union{A}{\Set{c}}$
                            be the committee.
                    \end{enumerate}
                        There are clearly $n$ ways to do the first step. Since $c$ is no
                        longer eligible to be in $A$ in the second step there are $\binom{n - 1}{k - 1}$
                        ways to do the second step. Thus by the rule of product we can conclude
                        that $|S| = n \binom{n - 1}{k - 1}$.
            \end{itemize}
            Thus we can conclude that $k \binom{n}{k} = n \binom{n - 1}{k - 1}$
            as required. \QED
        \end{proof}
    \section{Sterling Numbers}
        \begin{definition}
            Let $n \in \set{N}$ and $k \in \set{Z}$ such that $0 \le k \le n$. We define
            \npermk{n}{k}, read $n$ perm $k$, to be the number of ways to partitions
            an $n$-element set into $k$ cycles. Note that the cycle are treated as circular
            lists so the order of elements in each of the cycles is important. These numbers are
            called the Sterling Numbers of the First Kind.
        \end{definition}
        \begin{theorem}
            ``$\forall n \in \set{N} \ \forall k \in [n] \ \npermk{n}{k} = (n - 1) \npermk{n - 1}{k} + \npermk{n - 1}{k - 1}$.''
        \end{theorem}
        \begin{proof}
            Let $n \in \set{N}$ and $k \in [n]$ be given. We proceed via counting the set of all
            partitions of $[n]$ into $k$ cycles in two ways. We define this set as follows
            \[
                S = \Set{\Set{C_1, C_2, \dots, C_k} \ \mid \
                        \begin{array}{l}
                            \subs{\Set{P_1, P_2, \dots, P_k}}{\setsub{\power{[n]}}{\Set{}}}, \\
                            \forall i \in [k] \ C_i \mbox{ is a permutation of } P_i, \mbox{ and } \\
                            \dunion_{i = 1}^{n} P_i = [n]
                        \end{array}
                }
            \]
            \begin{itemize}
                \item
                    We can clearly see that $S$ has \npermk{n}{k} elements by definition
                    of the Sterling Numbers of the First Kind.
                \item
                    To count $S$ we partition it based on whether or not 1 is in it's
                    own cycle.
                    \begin{itemize}
                        \item
                            Since $1$ is in it's own cycle we now have to make $k - 1$
                            cycles out of $n - 1$ elements. There are \npermk{n - 1}{k - 1}
                            ways to do this by definition of the Sterling Numbers of
                            the Second Kind.
                        \item
                            If 1 is not in its own cycle then it must be in some other cycle.
                            We can make any element of $S$ where 1 is not in its own cycle
                            using the following 3 step process.
                            \begin{enumerate}
                                \item
                                    First take $\setsub{[n]}{\Set{1}}$ and partition it
                                    into $k$ cycles.
                                \item
                                    Pick one of these $k$ cycles to have 1 in it.
                                \item
                                    Now place $1$ among the elements in the chosen cycle.
                            \end{enumerate}
                            There are clear \npermk{n - 1}{k} ways to do step 1. There
                            are $n - 1$ ways to do steps 2 and 3. We can see this by noting
                            that this is equivalent to choosing an element of \Set{2, 3, \dots, n}
                            to have 1 come before it. We can now invoke the rule of product
                            to conclude that there are $(n - 1) \npermk{n - 1}{k}$ such partitions
                            of $[n]$ into cycles.
                    \end{itemize}
                    We can now invoke the rule of sum to conclude that  
                    $|S| = (n - 1) \npermk{n - 1}{k} + \npermk{n - 1}{k - 1}$
                    as required.
            \end{itemize}
            Thus we can conclude that $\npermk{n}{k} = (n - 1) \npermk{n - 1}{k} + \npermk{n - 1}{k - 1}$
            as required. \QED
        \end{proof}
        \begin{definition}
            Let $n \in \set{N}$ and $k \in \set{Z}$ such that $0 \le k \le n$. We define
            \npartk{n}{k}, read $n$ part $k$, to be the number of ways to partition
            an $n$ element set into $k$ parts. These numbers are called the Sterling Numbers
            of the Second Kind.
        \end{definition}
        \begin{theorem}
            ``$\forall n \in \set{N} \ \forall k \in [n] \ \npartk{n}{k} = k \npartk{n - 1}{k} + \npartk{n - 1}{k - 1}$.''
        \end{theorem}
        \begin{proof}
            Let $n \in \set{N}$ and $k \in [n]$ be given.
            We proceed via counting the set of all partitions of $[n]$ with $k$ parts
            in two ways. We define this set as follows
            \[
                S = \Set{\subs{\Set{P_1, P_2, \dots, P_k}}{\setsub{\power{[n]}}{\Set{}}} \ \mid \ \dunion_{i = 1}^{n} P_i = [n]}
            \]
            \begin{itemize}
                \item
                    We can clearly see that $S$ has \npartk{n}{k} elements by definition of
                    the Sterling Numbers of the Second Kind.
                \item
                    Now to count $S$ we partition it based on whether or not 1 is in
                    it's own part.
                    \begin{itemize}
                        \item
                            We know there are $\npartk{n - 1}{k - 1}$ partitions
                            where 1 is in its own part since after putting 1 in its own part we need
                            to make $k - 1$ more parts out of the remaining $n - 1$ elements.
                        \item
                            If 1 is not in its own part then it is in some other part. We
                            know there are exactly $k \npartk{n - 1}{k}$ partitions of this
                            kind because any element of this kind is produced by the following
                            two step process. First make partition the set $\Set{2, 3, \dots, n}$
                            into $k$ parts. Then pick which of these $k$ parts 1 will be put into.
                            The first step has $\npartk{n - 1}{k}$ options, while the second step
                            has $k$ options. Thus by the rule of product there are $k \npartk{n - 1}{k}$
                            such partitions.
                    \end{itemize}
                    Since this property partitions $S$ we can conclude that
                    $|S| = k \npartk{n - 1}{k} + \npartk{n - 1}{k - 1}$ via the rule of sum.
            \end{itemize}
            Thus we can conclude that $\npartk{n}{k} = k \npartk{n - 1}{k} + \npartk{n - 1}{k - 1}$
            as required. \QED
        \end{proof}
    \section{Cantor-Schroeder-Bernstein Theorem}
        \begin{definition}
            Let $A$ and $B$ be sets. We say that $A$ and $B$ have the same cardinality if and
            only if there is a bijection from $A$ to $B$.
        \end{definition}
        \begin{theorem}
            % TODO
        \end{theorem}
        \begin{proof}
            % TODO
        \end{proof}
    \section{Cantor's Diagonalization}
        \begin{definition}
            Let $S$ be a set. We say that $S$ is countably infinite if and only if
            there is a bijection from $\set{N}$ to $S$. We say that $S$ is uncountably
            infinite if and only if there is no bijection from $\set{N}$ to $S$ and
            $S$ is not finite.
        \end{definition}
        \begin{theorem}
            % $``\set{N}$ and $[0, 1)$ do not have the same cardinality.''
            $``\power{\set{N}}$ is uncountably infinite.''
            \label{diagonalization}
        \end{theorem}
        \begin{proof}
            Since $\power{\set{N}}$ is clearly infinite it suffices to show that there is
            no bijection from $\set{N}$ to $\power{\set{N}}$. Thus we proceed by showing
            that there cannot be a surjection from $\set{N}$ to $\power{\set{N}}$ via the
            method of diagonalization.

            Assume for the sake of contradiction that $\power{\set{N}}$ is countably infinite.
            Then there must be a function $f : \set{N} \rightarrow \power{\set{N}}$ such that
            $f$ is a bijection. By definition of bijectivity we know that $f$ must also be surjective.
            We will now construct $S \in \power{\set{N}}$ as follows.
            \[
                S = \Set{i \in \set{N} \mid i \not\in f(i)}
            \]
            We now proceed via contradiction to show that there is no $i \in \set{N}$ such
            that $f(i) = S$.
            Thus assume $\exists i \in \set{N}$ such that $f(i) = S$. We will now case on whether
            or not $i \in f(i)$.
            \begin{itemize}
                \item
                    If $i \in f(i)$, then by definition of $S$ we know that $i \not\in S$; thus $f(i) \neq S$.
                \item
                    If $i \not\in f(i)$, then by definition of $S$ we know that $i \in S$; thus $f(i) \neq S$.
            \end{itemize}
            In either case we can clearly see that $f(i) \neq S$. Thus we can conclude that
            $f$ is not a surjection (since it doesn't map anything to $S$), which is of course
            a contradiction. Thus $\power{\set{N}}$ is uncountably infinite. \QED

            % To show that $\set{N}$ and $[0, 1)$ do not have the same size it suffices
            % to show that there cannot be bijection from $\set{N}$ to $[0, 1)$. Thus we
            % will proceed by contradiction to show that there cannot be a surjection from
            % $\set{N}$ to $[0, 1)$.

            % Assume for the sake of contradiction that $\set{N}$ and $[0, 1)$ have the same
            % cardinality. By definition of cardinality we know that there must be some function
            % $f : \set{N} \rightarrow [0, 1)$ such that $f$ is a bijection. We now look at
            % the decimal expansion of $f(i)$ for each $i \in \set{N}$ and define
            % $A_{i, j}$ to be $j^{th}$ digit in the decimal expansion of $f(i)$. Thus we can
            % represent $f$ in tabular form as shown in \FigureRef{ftable}.
            % \begin{figure}[H]
            %     \centering
            %     $\begin{array}{c@{\ \mapsto \ }c@{\ . \ }ccccc}
            %         1 & 0 & \color{red} A_{1, 1} & A_{1, 2} & A_{1, 3} & A_{1, 4} & \cdots \\
            %         2 & 0 & A_{2, 1} & \color{red} A_{2, 2} & A_{2, 3} & A_{2, 4} & \cdots \\
            %         3 & 0 & A_{3, 1} & A_{3, 2} & \color{red} A_{3, 3} & A_{3, 4} & \cdots \\
            %         4 & 0 & A_{4, 1} & A_{4, 2} & A_{4, 3} & \color{red} A_{4, 4} & \cdots \\
            %         \vdots & \vdots & \vdots & \vdots & \vdots & \vdots & \ddots
            %     \end{array}$
            %     \caption{A tabular representation of $f$.}
            %     \label{ftable}
            % \end{figure}
            % We now construct $q \in [0, 1)$ as follows.
            % \begin{equation}
            %     q = \sum_{i = 1}^{\infty} B_i \cdot 10^{-i} = 0.B_1B_2B_3B_4\cdots \mbox{, where }
            %     B_{i} = \left\{\begin{array}{cl}
            %                 1 & \mbox{, if $A_{i, i} \neq 1$} \\
            %                 2 & \mbox{, otherwise}
            %             \end{array}\right.
            % \end{equation}
            % Since we know that every decimal expansion corresponds to a real number, we know that
            % $q \in [0, 1)$. Thus we should be able to find $j \in \set{N}$ such that
            % $f(j) = q$, but such a $j$ cannot exist because we know that $q$ will differ from $f(j)$
            % in the $j^{th}$ decimal place $\forall j \in \set{N}$ by construction of $q$.
            % Thus $f$ cannot be surjective since $q$ is a real number between 0 and 1 that is not
            % in the image of $f$. Thus $f$ is not a bijection, which is of course a contradiction.
            % \QED
        \end{proof}
        \begin{corollary}
            % TODO
        \end{corollary}
        \begin{proof}
            % TODO
        \end{proof}
        % \begin{corollary}
        %     $``\set{N}$ and $\set{R}$ do not have the same cardinality.''
        % \end{corollary}
        % \begin{proof}
        %     By \TheoremRef{diagonalization} we know that $\set{N}$ and $[0, 1)$ do not
        %     have the same cardinality. Since $\subs{[0, 1)}{\set{R}}$ and we know that $[0, 1)$
        %     is larger than $\set{N}$ in some sense, we can conclude that $\set{N}$
        %     and $\set{R}$ do not have the same cardinality. \QED
        % \end{proof}
    \section{Inclusion Exclusion Theorem}
        \begin{theorem}
            % TODO
        \end{theorem}
        \begin{proof}
            % TODO
        \end{proof}
    \section{Generating Functions}
        \begin{definition}
            We define the ordinary power series generating function of a sequence of integers
            $a_0, a_1, \dots$ to be $A(x) = \dsum_{n = 0}^{\infty} a_n x^n$.
        \end{definition}
        \begin{theorem}
            ``The generating function of a sequence is rational if and only if the sequence is linear recurrent.''
        \end{theorem}
        \begin{proof}
            We proceed by showing each direction of the claim separately.
            \begin{itemize}
                \item
                    Consider the following linear $k^{th}$ order linear recurrence.
                    \begin{equation}
                        a_{n + k}= \sum_{i = 0}^{k - 1} c_i a_{n + i} \mbox{, where $\forall i \in \Set{0, \dots, k - 1} c_i \in \set{Z}$}
                    \end{equation}
                    We will now try to find the ordinary power series generating function of this recurrence
                    as follows
                    \begin{equation}
                        A(x) = \sum_{n = 0}^{\infty} a_n x^n
                    \end{equation}
                    We first define the generating function we are looking for above.
                    \[\begin{array}{cr@{\ = \ }l}
                              & \dsum_{n = 0}^{\infty} a_{n + k} x^n  & \dsum_{n = 0}^{\infty} \dsum_{i = 0}^{k - 1} c_i a_{n + i} x^n \\
                        \then & \dfrac{1}{x^k} \dsum_{n = 0}^{\infty} a_{n + k} x^{n + k} & \dsum_{i = 0}^{k - 1} \dsum_{n = 0}^{\infty} c_i a_{n + i} x^n \\
                        \then & \dfrac{1}{x^k} \left(\dsum_{n = 0}^{\infty} a_n x^n - \dsum_{n = 0}^{k - 1} a_n x^n\right) & \dsum_{i = 0}^{k - 1} c_i \dsum_{n = 0}^{\infty} a_{n + i} x^n \\
                        \then & \dfrac{1}{x^k} \left(A(x) - \dsum_{n = 0}^{k - 1} a_n x^n\right) & \dsum_{i = 0}^{k - 1} \dfrac{c_i}{x^i} \dsum_{n = 0}^{\infty} a_{n + i} x^{n + i} \\
                        \then & \dfrac{1}{x^k} \left(A(x) - \dsum_{n = 0}^{k - 1} a_n x^n\right) & \dsum_{i = 0}^{k - 1} \dfrac{c_i}{x^i} \left(\dsum_{n = 0}^{\infty} a_{n} x^{n} - \dsum_{n = 0}^{i - 1} a_n x^n\right) \\
                        \then & A(x) - \dsum_{n = 0}^{k - 1} a_n x^n & \dsum_{i = 0}^{k - 1} c_i x^{k - i} \left(A(x) - \dsum_{n = 0}^{i - 1} a_n x^n\right) \\
                        \then & A(x) - \dsum_{n = 0}^{k - 1} a_n x^n & \dsum_{i = 0}^{k - 1} c_i x^{k - i} A(x) - \dsum_{i = 0}^{k - 1} c_i x^{k - i} \dsum_{n = 0}^{i - 1} a_n x^n \\
                        \then & A(x) - \dsum_{i = 0}^{k - 1} c_i x^{k - i} A(x) & \dsum_{n = 0}^{k - 1} a_n x^n - \dsum_{i = 0}^{k - 1} c_i x^{k - i} \dsum_{n = 0}^{i - 1} a_n x^n \\
                        \then & A(x) \left(1 - \dsum_{i = 0}^{k - 1} c_i x^{k - i}\right) & \dsum_{n = 0}^{k - 1} a_n x^n - \dsum_{i = 0}^{k - 1} c_i x^{k - i} \dsum_{n = 0}^{i - 1} a_n x^n \\
                        \then & A(x) \left(1 - \dsum_{i = 0}^{k - 1} c_i x^{k - i}\right) & \dsum_{n = 0}^{k - 1} \left(a_n - \dsum_{i = k - n}^{k - 1} c_i a_{n - (k - j)}\right) x^n\\
                    \end{array}\]
                    We can now define $b_n$ for $0 \le n \le k - 1$ as follows:
                    \begin{equation}
                        b_n = \left\{\begin{array}{ll}
                                  a_0 & \mbox{, if $n = 0$} \\
                                  a_n - \dsum_{i = k - n}^{k - 1} c_i a_{n - (k - j)} & \mbox{, otherwise}
                              \end{array}\right.
                    \end{equation}
                    We can now write $A(x)$ is the following convenient form
                    \begin{equation}
                        A(x) = \frac{\dsum_{n = 0}^{k - 1} b_n x^n}{1 - \dsum_{i = 0}^{k - 1} c_i x^{k - i}}
                    \end{equation}
                    We can clearly see that $A(x)$ is a rational function. Thus every linear recurrence has a rational generating function.
                \item
                    We now proceed to show that every rational generating function generates a linear recurrent sequence of integers.
                    So consider the following rational generating function
                    \begin{equation}
                        A(x) = \frac{f(x)}{g(x)} = \sum_{n = 0}^{\infty} a_n x^n
                    \end{equation}
                    Where $f(x)$ and $g(x)$ are polynomials of degree $j$ and $k$ respectively, i.e.
                    \[
                        f(x) = \sum_{n = 0}^{d_1} f_n x^n \mbox{ and } g(x) = \sum_{n = 0}^{d_n} g_n x^n
                    \]
                    We now try to find a linear recurrence for the sequence that is generated by $A(X)$
                    as follows:
                    \[\begin{array}{rr@{\ = \ }l}
                              & \dfrac{f(x)}{g(x)} & \dsum_{n = 0}^{\infty} a_n x^n\\
                        \then & f(x) & g(x) \dsum_{n = 0}^{\infty} a_n x^n \\
                        \then & \dsum_{n = 0}^{j} f_n x^n & \dsum_{i = 0}^{k} g_i x^i \dsum_{n = 0}^{\infty} a_n x^n \\
                        \then & \dsum_{n = 0}^{j} f_n x^n & \dsum_{i = 0}^{k} \dsum_{n = 0}^{\infty} g_i a_n x^{n + i} \\
                    \end{array}\]
                    We now want to group together like factors of $x^i$ on the right hand side by splitting up the
                    sum into two sums as follows.
                    \[\begin{array}{rr@{\ = \ }l}
                        \then & \dsum_{n = 0}^{j} f_n x^n & \left(\dsum_{i = 0}^{k - 1} \dsum_{l = 0}^{i} g_i a_{i - l} x^i\right) + \left(\dsum_{n = 0}^{\infty} \dsum_{i = 0}^{k} g_i a_{n + i} x^{n + k}\right)
                    \end{array}\]
                    The two summations on the right-hand side of the above equation represent the initial conditions and recurrence relation
                    of the linear recurrence represented by $A(x)$. We now proceed to solve for the initial conditions and linear
                    recurrence by using the fact that two ordinary power series are equal if and only if their coefficients are equal.
                    Since we know the $f(x)$ is a polynomial of order $j$, and that $j < k$ we can conclude that
                    \[
                        \forall n \in \union{\set{N}}{\Set{0}} \ \sum_{i = 0}^{k} g_i a_{n + i} = 0
                    \]
                    We can solve the above equation of $a_{n + k}$ in terms of $a_n, \dots, a_{n + k - 1}$ and conclude that
                    \begin{equation}
                            a_{n + k} = \sum_{i = 0}^{k - 1} -\frac{g_i}{g_0} a_{n + i}
                        \label{recurrence equation}
                    \end{equation}
                    Thus we can clearly see that $a_i$'s can be generated by a linear recurrence of order $k$. All that's left now is to determine
                    the initial conditions. We again use the fact that two ordinary power series are equal if and only if they
                    have the same coefficients to conclude that
                    \begin{equation}
                        \forall 0 \le i < k \ \sum_{l = 0}^{i} g_i a_{i - l} = f_i
                        \label{initial conditions equations}
                    \end{equation}
                    By solving this system of equations we can determine the initial conditions of our linear recurrence since
                    we have $k$ equation and $k$ unknowns. Thus we can conclude that rational generating functions generate
                    linear recurrences.
            \end{itemize}
            Thus we can conclude that the generating function of a sequence is
            rational if and only if the sequence is linear recurrent. \QED
        \end{proof}

